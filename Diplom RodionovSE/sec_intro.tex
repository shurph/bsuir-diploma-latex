\sectioncentered*{Введение}
\addcontentsline{toc}{section}{Введение}
\label{sec:intro}

Темп современной жизни не позволяет расслабиться, требуя от нас с каждым днем всё больше. Мы начинаем опаздывать, паниковать и искать панацею, способную мгновенно решить все проблемы. Обращаясь к тайм-менеджменту и начиная внедрять его приемы на практике, мы ожидаем быстрых и волшебных результатов. Если таковые не появляются, обвиняем систему в несостоятельности и бросаемся искать другую.

Прогрессивной мир задает жизненный темп в котором человеку попросту не хватает времени чтобы завершить все свои дела, требуя от него все больше с каждым днем. Он начинает опаздывать, паниковать, оставляя часть своих дел не решенными. Одним из решений данной проблемы является более серьозный подход к организации времени(англ. "тайм-менеджмент"). Обращаясь к тайм-менеджменту и начиная внедрять его приемы на практике, мы ожидаем быстрых и волшебных результатов. Если таковые не появляются, обвиняем систему в несостоятельности и бросаемся искать другую.

Тайм-менеджмент включает в себя широкий спектр деятельности, в числе которых:
\begin{itemize}
	\item постановка задач;
	\item разбиение задач на подзадачи;
	\item приоритезация задач;
	\item анализ потраченного времени;
\end{itemize}

Одним из наиболее удобных и простых инструментов тайм-менеджмента является создание «To-do list» – списка того, что нужно сделать. Это перечень задач на день, неделю или другой четко очерченный период времени. Он составляется в хаотичном порядке, а затем могут быть расставлены приоритеты.
Дневной список «To-do list» может состоять из:
\begin{itemize}
	\item встреч;
	\item перемещения по городу;
	\item походы в магазин;
	\item важный звонки;
	\item самообразование;
\end{itemize}

\ignore{%
Источник: http://timestep.ru/2010/01/15/idealnaya-sistema-tajjm-menedzhmenta#ixzz4bnydNl2Z
}

