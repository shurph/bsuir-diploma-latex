\sectioncentered*{Введение}
\addcontentsline{toc}{section}{Введение}
\label{sec:intro}

Темп современной жизни не позволяет расслабиться, требуя от нас с каждым днем всё больше. Мы начинаем опаздывать, паниковать и искать панацею, способную мгновенно решить все проблемы. Обращаясь к тайм-менеджменту и начиная внедрять его приемы на практике, мы ожидаем быстрых и волшебных результатов. Если таковые не появляются, обвиняем систему в несостоятельности и бросаемся искать другую.


Но существует ли идеальная методика тайм-менеджмента, подходящая всем: и менеджерам среднего звена, и руководителям крупных корпораций, и молодым мамам, и студентам, и фрилансерам?

Современных систем управления временем достаточно много. Стивен Кови, международный бизнес-тренер и автор нескольких трудов по лайф-менеджменту, разделил все методики тайм-менеджмента на 4 поколения:

Первое сконцентрировано на составлении перечней и памяток;
Второе — на составлении планов и подготовке к их реализации;
Третье — на составлении планов, расстановке акцентов и контроле их выполнения.
Четвертое — основано на жизненных принципах и новом взгляде на жизнь в целом.
По мнению Стивена Кови, первые три методики давно не актуальны, хотя, как показывает практика, они все еще активно используются.

\ignore{%
Источник: http://timestep.ru/2010/01/15/idealnaya-sistema-tajjm-menedzhmenta#ixzz4bnydNl2Z
}

