\subsection{Node}

Чуть более недели назад на хабре появилась статья, в которой затрагивалась «проблема»: Node.js — это JavaScript или нет. Некоторые аргументы, представленные в статье были справедливыми, но, увы, безосновательны. Другие же аргументы были вовсе абсурдными и не правдивыми. Я не буду писать о знаниях автора статьи в данной области, даже не буду давать ссылки на это статью (дабы статья перенесена в черновики, она осталась только в архивах). Я же просто сравню скрипты Node.js и JavaScript в таком виде, в котором все его привыкли видеть.

Введение

Для начала обратимся к Википедии и узнаем, что есть такое Node.js и JavaScript:
Node или Node.js — серверная реализация языка программирования JavaScript, основанная на движке V8. Предназначена для создания масштабируемых распределённых сетевых приложений, таких как веб-сервер. Node.js по целям использования сходен с каркасами Twisted на языке Python и EventMachine на Ruby. В отличие от большинства программ JavaScript, этот каркас исполняется не в браузере клиента, а на стороне сервера.
JavaScript — прототипно-ориентированный сценарный язык программирования. Является диалектом языка ECMAScript.

Что ж, определение Node.js немного расплывчато, и надо сказать, не корректно. Тогда посмотрим на информацию на официальном сайте:
Node.js is a platform built on Chrome's JavaScript runtime for easily building fast, scalable network applications. Node.js uses an event-driven, non-blocking I/O model that makes it lightweight and efficient, perfect for data-intensive real-time applications that run across distributed devices.

Самое главное слово здесь — платформа. Оно и характеризует весь Node.js. Из всего вышесказанного можно сделать предварительный вывод, что Node.js это среда выполнения JavaScript, точно как браузер, с той лишь разницей, что у нас нет доступа к DOM (а собственно, зачем он нужен на стороне сервера?; однако существует библиотека для работы с DOM — jsdom).
