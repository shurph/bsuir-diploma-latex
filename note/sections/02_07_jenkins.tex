\subsection{Jenkins}
\label{sec:analysis:jenkins}

Jenkins – это инструмент непрерывной интеграции, который чаще всего используется для разработки программного обеспечения. Это среда автоматизации, которая выполняет повторяющиеся задания. Jenkins может выполнять и контролировать выполнение команд на удаленных системах, а также всего того, что можно выполнить из командной строки. 

Непрерывная интеграция (continuous integration) — это очень, очень хорошо. Вы настраиваете ее один раз, и все все процессы интеграции происходят без вашего участия. Некоторые плюсы использования Jenkins:
\begin{itemize}
  \item когда кто-то ломает билд, вы узнаете об этом сразу, что позволяет быстро устранить проблему.
  \item вы можете автоматизировать прогон тестов, деплой приложения на тестовый сервер, проверку code style и тому подобные вещи.
  \item также в Jenkins можно хранить собранные deb-пакеты, отчеты о прогоне тестов или Javadoc/Doxygen/EDoc-документацию. 
\end{itemize}
