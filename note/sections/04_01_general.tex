\subsection{Общие требования}


\begin{itemize}
  \item[1] Назначение разработки.

\hspace*{2.5em}Итогом дипломного проектирования должен быть проект, позволяющий пользователям управлять своими серверами через web-интерфейс. 
Продукт должен быть интересен большому и среднему бизнесу, который хочет сократить время на администрирование своих серверов.
Возможность строить графики различной сложности для удобного и быстрого анализа.

  \item[2] Входные данные.

\hspace*{2.5em}Входными данными являются данные с серверов. Они получаются с помощью софта, установленного на сервера клиента.
При создании графиков на сайте, пользователь должен иметь возможность указывать интересующие его параметры событий, по этим параметрам строятся графики.

  \item[3] Выходные данные.

\hspace*{2.5em}Выходными данными можно считать информацию, которую получают владельцы приложений-потребителей о всех действия и всей информации с серверов. Функционал, позволяющий строить разнообразные графики, позволяющий анализировать информацию и сообщать о нарушении работы. 
 
  \item[4] Требования к временным характеристикам.

\hspace*{2.5em}Сохранение событий должно происходить быстро и не вызывать задержек в приложении-потребителе, так как это может вызывать недовольство и желание отказаться от текущего продукта. 

  \item[5] Требования к надежности.

\hspace*{2.5em}Система должна работать без остановок, так как сервера имеют большую важность для клиента.

  \item[6] Требования к составу и параметрам технических и программных средств.

\hspace*{2.5em}Node для сервера.
Angular для клиента.
nxinx + docker - веб-сервера для поддержки клиента.

\end{itemize}
 