\subsection{Общие требования}

Назначение разработки
Итогом дипломного проектирования должен быть проект, позволяющий пользователям управлять своими серверами через web-интерфейс. 
Продукт должен быть интересен большому и среднему бизнесу, который хочет сократить время на администрирования своих серверов.
Возможность строить графики различной сложности для удобного и быстрого анализа.
 
Входные данные
Входными данными являются данные с серверов. Они получаются с помощью софта установленного на сервера клиента.
При создании графиков на сайте, пользователь должен иметь возможность указывать интересующие его параметры событий, по этим параметрам строятся графики.
 
Выходные данные
Выходными данными можно считать информацию, которую получают владельцы приложений-потребителей о всех действия и всей информации с серверов. Функционал позволяющий строить разнообразные графики, позволяющий анализировать информацию и сообщать о нарушении работы. 
 
Требования к временным характеристикам
Сохранение событий должно происходить быстро и не вызывать задержек в приложении-потребителе, так как это может вызывать недовольство и желание отказаться от текущего продукта.
 
Требования к надежности
Система должна работать без остановок, так как сервера имеют большую важность для клиента.
 
Требования к составу и параметрам технических и программных средств
Node для сервера.
Angular для клиента.
nxinx + docker - веб-сервера для поддержки клиента.
