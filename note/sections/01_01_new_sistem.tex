\subsection{Необходимость разработки сервиса}

Стек технологий как и требования к серверам с каждым днем растут. Появляются сотни инструментов для разработки и администрирования серверов. И в этом многообразии бывает сложно разобраться, а еще нужно следить за логами, базай данных и конфигурациями. Один из путей решения придумать какую-то корпоративный сервис для интересующих задач. Но это является затратным и долгим процессом. Тем более, если это небольшая фирма, где доходы не такие большие.

Еще одним решением этой задачи является использование уже готовых инструментов и сервисов. Но здесь тоже не все так хорошо. Некоторые сервисы являются очень дорогими, некоторые не являются достаточно удобными, но главный минус сторниих сервисов — они предоставляют ускоспецелизированный функционал. Т.е. чтобы решить все задачи придется пользоваться большим количеством сервисов и в случае отказа одного, будет нарушаться работа, что сопровождается потеряй денег.

Разработка и применение программного продукта “Сервис управления серверами” позволит отказаться от имеющихся аналогов и реализовать
систему, которая поддерживает современный стек технологий в одном месте. По сравнения с первым путем это будет экономичнее для компании, по сравнению со второй – увеличивается надежность и удобство, так как чтобы просмотреть логи, системную информаци, поменять конфиг, открыть консоль нужного сервера и многое другое будет достаточно нескольких кликов мышки.

