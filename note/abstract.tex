\sectioncentered*{Реферат}
\thispagestyle{empty}

% Добавляем три к счетчику страниц, поскольку двухстороннее задание, а также ведомость печатаются отдельно
% \FPeval{\totalpagesnumber}{round(\arabic{pagesLTS.pagenr} + 3, 0)}
% \newcommand{\totalpages}{\num{\totalpagesnumber}} 

Дипломный проект выполнен на 6 листах (3 чертежа и 3 плаката) с пояснительной запиской на 75 страницах, содержит 28 рисунков. Использовано 10 источников литературы.
% \begin{center}
	% Пояснительная записка \totalpages~с., \num{\totfig{}}~рис., \num{\tottab{}}~табл., \num{\toteq{}}~формул, \num{\totref{}}~источника.

Ключевые слова:
\MakeUppercase{Программное средство, веб-приложение, университет, организация учебного процесса, расписание, индивидуальные задания}
% \end{center}

Целью данного проекта является разработка сервиса, который предназначем для пользователей, у которых есть необходимость следить и управлять большим количеством серверов. Пояснительная записка к дипломному проекту детально описывает процесс разработки
вышеуказанного программного комплекса и включает 8 разделов.

В первом разделе данного проекта описана постановка задачи перед
разработчиком проекта.

Во втором разделе анализируются предметная область. Затрагивается
современные технологии и решения для разработки и администрирования серверов, чтобы понять какой функционал должен предоставлять сервис.

В третьем разделе анализируются существующие решения.

В четвертом разделе формулируются функциональные требования для
проекта, а также к входным и выходным данным, техническим средствам.

В пятом разделе описываются инструменты разработки, которые были
использованы для реализации проекта.

В шестом разделе описано проектирование системы, созданы схемы и  диаграммы для лучшего представления некаторых аспектов разработки.

В седьмом разделе рассмотрены некоторые вопросы, связанные
непосредственно с реализацией проекта.

В восьмом разделе приведено технико-экономическое обоснование
эффективности разработки и использования программного продукта.