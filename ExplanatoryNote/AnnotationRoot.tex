\input{Preamble}
\newcommand{\DotNet}{.NET Frame\-work\xspace}
\newcommand{\CSharp}{C\#\xspace}
\newcommand{\CPP}{C++\xspace}

\newcommand{\BinaryWars}{<<Bi\-na\-ry Wars>>\xspace}
\newcommand{\TicTacToe}{<<Кре\-сти\-ки-но\-ли\-ки>>\xspace}

\begin{document}

\sectioncentered*{Аннотация}
\thispagestyle{empty}

\begin{center}
    \begin{minipage}{0.84\textwidth}
        на дипломный проект <<Игра Bi\-na\-ry Wars>> студента УО <<Белорусский государственный университет информатики и радиоэлектроники>> Козлова~Д.\,С.
    \end{minipage}
\end{center}

\emph{Ключевые слова}: программное средство, компьютерная игра, трёхмерная и двухмерная графика, кроссплатформенная многопользовательская игра, теория игр.

\vspace{1\parsep}

Цель настоящего дипломного проекта состоит в разработке компьютерной игры \BinaryWars. Игра представляет собой продвинутый аналог игры \TicTacToe со следующими особенностями: различные игровые режимы (двухмерное игровое поле три на три клетки, трёхмерное игровое поле), возможность играть против компьютерного противника, кроссплатформенная многопользовательская игра со случайным или определённым оппонентом, наличие внутриигрового магазина.

Первый раздел содержит обзор предметной области по теме дипломного проекта, аналоги создаваемого программного продукта, анализ их достоинств и недостатков. На основе проведенного анализа и с учетом требований формулируются требования к проектируемому программному продукту.

Во втором разделе сформулированы функциональные требования для проектируемого программного продукта.

В третьем разделе производится краткий обзор технологий, использованных для реализации программного продукта.

Четвёртый раздел посвящен проектированию архитектуры приложения. В нём рассмотрены различные архитектурные подходы и шаблоны проектирования, которые нашли применение при реализации программного продукта.

В пятом разделе дано описание процесса разработки программного продукта в рамках дипломного проекта.

В шестом разделе приведено технико-экономическое обоснование эффективности разработки программного продукта.

В заключении подводятся итоги и делаются выводы по дипломному проекту, а также описывается дальнейший план развития проекта.

Дипломный проект выполнен самостоятельно, проверен в системе <<Антиплагиат>>. Процент оригинальности соответствует норме, установленной кафедрой информатики. Цитирования обозначены ссылками на публикации, указанные в <<Списке литературы>>.

\end{document}