\newpage
\phantomsection
\addcontentsline{toc}{section}{Введение}
\sectioncentered*{Введение}

На сегодняшний день рынок развлечений является одним из самых прибыльных. С момента начала информационно-технической революции мир стремительно движется в будущее, создавая все более совершенные компьютерные системы, чтобы облегчить жизнь человека, а также занять его досуг.

С появлением персональных компьютеров, с каждым годом, их роль в жизни людей постоянно растет. Компьютер стал незаменимым помощником не только в сфере экономических расчетов, но и является мощным центром развлечений. Сферы влияния кино и литературы ощущают мощное давление со стороны интерактивных развлечений, устройств дополненной реальности и прочих, внедряемых благодаря развитию компьютерных технологий. Современный человек иногда даже не задумывается, что и его смартфон является аналогичным мобильным персональным компьютером. С ростом роли компьютеров в жизни человека, компьютерная техника существенно влияет и на модель поведения человека. По последним исследования средний возраст игрока компьютерных игр составляет 30 лет и выше.

Компьютерные развлечения делают жизнь человека богаче, насыщеннее и как следствие -- это мощная экономическая сфера приносящая огромные доходы.
Поэтому не случайно, что особая роль в жизни современного человека отводится компьютерным играм, первые из которых существовали на самой заре компьютерной технике.

Можно сказать, что игры -- это определенный вид искусства, схожий с другими зрелищными жанрами. Игры могут нести не только развлекательный характер, но и заставлять задуматься, переживать, поднимать серьезные глобальные или психологические вопросы. Другими словами -- игры -- это современный вид искусства. Компьютерные игры порой дарят эмоций не меньше, чем просмотр кинофильма или театральной постановки.

Над современными играми работают огромные коллективы программистов, сценарии к играм пишут профессиональные писатели. Компьютерные игры стали важной экономической составляющей. Доходы игровой индустрии за 2016 год достигли 91 миллиарда долларов \cite{GameDevRevenues}.

Но в мире востребованы не только высокобюджетные компьютерные игры. Многие программисты-любители -- инди-разработчики создают небольшие игровые программы, которые не обладают современной дорогостоящей высококлассной графикой, как игры популярных компаний, но зачастую либо имеют инновационный игровой процесс, либо обладают интересным сюжетом и т.д., и в итоге они так же пользуются большой популярностью и приносят создателям достаточные для жизни доходы.

В мире существует чрезвычайно большое количество любителей компьютерных игр с разным опытом в этой сфере. Есть начинающие любители игр, есть ностальгирующие опытные игроки, поэтому на этом рынке востребованы игровые проекты разной направленности.

Данная дипломная работа ставит целью создание компьютерной игры для мобильных устройств под управлением операционных систем Android и iOS, представляющей собой продвинутый аналог игры \TicTacToe со следующими особенностями: различные игровые режимы (игровое поле 3 на 3 клетки, трёхмерное игровое поле), возможность играть против компьютерного противника с выбором уровня сложности, кроссплатформенная многопользовательская игра со случайным или определённым оппонентом, наличие внутриигрового магазина.

В рамках дипломного проекта реализуется множество отдельных слабосвязанных компонентов (система UI, сетевая система, внутриигровой магазин, игровая логика, компьютерный противник, механизм внедрения зависимостей, аудио система и т.д.), на основе которых строится компьютерная игра.