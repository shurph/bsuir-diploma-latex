\sectioncentered*{Реферат}
\thispagestyle{empty}

% Зачем: чтобы можно было вывести общее число страниц.
% Добавляется единица, поскольку последняя страница -- ведомость.
\FPeval{\totalpages}{round(\getpagerefnumber{LastPage} + 1, 0)}

\MakeUppercase{Игра \BinaryWars :} дипломный проект / Д.\,С.~Козлов. -- Минск~: БГУИР, 2017, -- п.з. -- \num{\totalpages} с., чертежей (плакатов) -- \num{6} л. формата А1.

\vspace{1\parsep}

Пояснительная записка \num{\totalpages} с., \num{\totfig} рис., \num{\tottab} табл., \num{\toteq} формул, \num{\totref} источников.

\vspace{1\parsep}

\MakeUppercase{Программное средство, компьютерная игра, трёхмерная и двухмерная графика, кроссплатформенная многопользовательская игра, теория игр}

\vspace{1\parsep}

Цель настоящего дипломного проекта состоит в разработке компьютерной игры \BinaryWars. Игра представляет собой продвинутый аналог игры \TicTacToe со следующими особенностями: различные игровые режимы (двухмерное игровое поле три на три клетки, трёхмерное игровое поле), возможность играть против компьютерного соперника с выбором уровня сложности, кроссплатформенная многопользовательская игра со случайным или определённым оппонентом, наличие внутриигрового магазина.

В процессе анализа предметной области были рассмотрены существующие на рынке аналоги, определены их достоинства и недостатки. Выработаны функциональные и нефункциональные требования.

Была разработана архитектура программной системы, для каждой ее составной части было проведено разграничение реализуемых задач, проектирование, уточнение используемых технологий и собственно разработка. Были выбраны наиболее современные средства разработки, широко применяемые в индустрии. Произведены исследования в области алгоритмов поиска для реализации игры против компьютерного соперника с учётом ограниченности в ресурсах целевых устройств (мобильные устройства под управлением операционных систем Android и iOS).

Полученные в ходе технико-экономического обоснования результаты о прибыли для разработчика, уровень рентабельности, а также экономический эффект доказывают целесообразность разработки проекта.