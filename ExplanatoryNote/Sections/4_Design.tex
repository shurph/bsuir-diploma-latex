\section{Проектирование программного средства}

Компьютерные игры бывают совершенно разных видов и соответственно требования к ним по архитектуре могут совершенно разнится. С одной стороны, компьютерные игры являются очень динамичными и разногранными системами, что требует продуманной архитектуры для обеспечения максимальной гибкости кода. С другой стороны, в отличии от многих других программных продуктов требования к производительности в компьютерных играх очень высоки -- они должны самым оптимальным способом использовать вычислительную мощь как центрального процессора, так и видеокарты, что также нужно учитывать. Исходя из этих факторов можно сказать, что проектирование архитектуры в компьютерных играх не является тривиальной задачей и от программиста в данном случае требуется большой багаж знаний и опыта в самых различных областях информатики.

Современные игровые движки стараются выполнять максимальное количество сложной и требовательной к производительности работы за программиста, но тем не менее, игровые движки являются инструментом, призванным облегчить работу, и только от самого программиста зависит то, насколько эффективно он сможет использовать этот инструмент.

В данном разделе будут рассмотренные основные архитектурные подходы и шаблоны проектирования, применение которых нашло место при создании программного продукта. Условно все их можно подразделить на те, которые повышают гибкость, уменьшают связанность в коде и те, которые призваны повысить производительность программного продукта. Также в этом разделе будет произведено описание архитектуры всего создаваемого программного продукта.


\subsection{Шаблоны игрового программирования}


\subsubsection{Игровой цикл}

Задача данного шаблона игрового программирования состоит в том, чтобы устранить зависимость игрового времени от пользовательского ввода и скорости процессора.

Игровой цикл -- это по сути сама сущность примера <<шаблона в игровом программировании>>. Он есть практически в каждой игре и двух одинаковых практически нет. И при этом в не играх он встречается крайне редко~\cite{GameProgrammingPatterns}.

Современные программы с графическим пользовательским интерфейсом обычно просто ничего не делают до тех пор, пользователь не произведёт какой-либо ввод:

\begin{lstlisting}[caption={Пример цикла ожидания ввода}]
while(true)
{
    Event event = WaitForEvent();
    DispatchEvent(event);
}
\end{lstlisting}

Программа блокируется в ожидании пользовательского ввода, что на самом деле является проблемой.

В отличие от других программ, игры продолжают работать даже когда пользователь не предоставляет никакого ввода. Если вы будете просто смотреть на экран, игра не остановится. На основании этого можно выделить ключевую особенность игрового цикла: он обрабатывает пользовательский ввод, но не ожидает его. Цикл продолжает выполняться всегда:

\begin{lstlisting}[caption={Пример простого игрового цикла}, label=Listing:Design:SimpleLoop]
while(true)
{
    ProcessInput();
    Update();
    Render();
}
\end{lstlisting}

Игровой цикл работает на протяжении всей игры. На каждом своем цикле игровой цикл обрабатывает пользовательский ввод, обновляет состояние игры и визуализирует игру. А еще он следит за ходом времени и управляет скоростью игрового процесса.

Простой игровой цикл, как на листинге \ref{Listing:Design:SimpleLoop} не всегда подходит: на быстрой машине он будет работать так быстро что пользователь даже не разберет что происходит. На медленной машине игра будет просто тормозить. Если в какой-то части игры у вас есть сложный AI или физика, игра тоже будет замедляться в этих местах. Для решения этой проблемы добавляют некоторую паузу в конце цикла обновления путём задания максимального количества кадров в секунду -- FPS, и на её основе вычисляют необходимую паузу:

\begin{lstlisting}[caption={Пример игрового цикла с паузой}, label=Listing:Design:PauseLoop]
while(true)
{
    double start = GetCurrentTime();
    ProcessInput();
    Update();
    Render();

    Sleep(start + MS_PER_FRAME - GetCurrentTime());
}
\end{lstlisting}

Но создание игрового цикла так же не ограничивается циклом, показанным на листинге \ref{Listing:Design:PauseLoop}. На деле при создании компьютерных игр циклы могут иметь совершенно различные виды, которые можно разбить на следующие категории:
\begin{enumerate}
    \item контроль над игровым циклом:
    \begin{enumerate}
        \item использование цикла событий платформы;
        \item использование игрового цикла движка;
        \item написание самостоятельно;
    \end{enumerate}
    \item потребление энергии:
    \begin{enumerate}
        \item работа с максимальной скоростью;
        \item ограничение количества кадров в секунду;
    \end{enumerate}
    \item управление скоростью игры:
    \begin{enumerate}
        \item фиксированный временной шаг без синхронизации;
        \item фиксированные временные шаги с синхронизацией;
        \item переменная длительность временного шага;
        \item фиксированный временной шаг обновления и непостоянный рендеринг.
    \end{enumerate}
\end{enumerate}

Но этот шаблон не такой как другие. С уверенностью можно сказать что он используется во всех играх. Даже если используется готовый движок, вы его не пишете сами, но он все равно присутствует.


\subsubsection{Метод обновления}

Задачей данного шаблона игрового программирования является симуляция коллекции независимых объектов с помощью указания каждому объекту обработки одного кадра поведения за раз.

Данный шаблон игрового проектирование применяется в связке с шаблоном <<Игровой цикл>>: для того, чтобы отделить специфичную игровую логику от кода игрового цикла и обеспечить инкапсуляцию поведения игровой сущности как правило создаётся уровень абстракции, путём определения специального интерфейса:

\begin{lstlisting}[caption={Интерфейс для метода обновления}, label=Listing:Design:UpdateInterface]
interface IUpdate
{
    void Update();
}
\end{lstlisting}

Игровой цикл поддерживает коллекцию объектов, реализующих интерфейс \ref{Listing:Design:UpdateInterface} но не знает их конкретный тип. Все что о них нужно знать -- это то, что их нужно обновлять. Таким образом мы отделяем поведение каждого объекта от игрового цикла и других объектов.

На каждом кадре игровой цикл проходит по всей коллекции и вызывает Update() для каждого объекта. Таким образом каждый объект может обновить свое поведение на один кадр. Вызывая его для объектов на каждом кадре, мы получаем их одновременное действие.

Игровой мир содержит коллекцию объектов. Каждый объект реализует метод обновления, симулирующий один кадр поведения объекта. На каждом кадре игра обновляет каждый объект из коллекции.

Метод обновления хорошо работает, когда:
\begin{itemize}
    \item в игре есть некоторое количество объектов или систем, которые должны работать одновременно;
    \item поведение каждого объекта практически не зависит от остальных;
    \item объекты нужно обновлять постоянно.
\end{itemize}


\subsubsection{Компонент}

Данный шаблон игрового программирования позволяет одной сущности охватывать несколько областей, не связывая их между собой.

Когда объектно-ориентированное программирование появилось впервые, наследование было самым любимым из всех его инструментов. Оно было объявлено ультимативным молотом повторного использования кода. С тех пор мы на собственных ошибках убедились в том, что этот молот может быть слишком тяжел. Наследование имеет свое применение, но для повторного использования кода оно обычно слишком громоздко.

Ему на замену в программирование пришел новый тренд: композиция взамен наследования везде, где это возможно. Вместо совместного использования кода двумя классами, которые наследуются от какого-то одного класса, мы позволяем им обеим обладать одним и тем же экземпляром этого класса.

Единая сущность охватывает множество областей. Для сохранения изолированности областей, код для каждой помещается в свой собственный класс компонент. Сущность упрощается до простого контейнера компонентов.

Компоненты чаще всего можно найти внутри класса-ядра, описывающего сущности в игре, но использовать их можно и в других частях игры. Этот шаблон стоит использовать если верно что-либо из нижеперечисленного:
\begin{itemize}
    \item есть класс, затрагивающий множество областей, которые вы хотите оставить несвязанными друг с другом;
    \item класс становится слишком массивным и сложным для работы;
    \item необходимо определить множество объектов, разделяющих различные возможности, но при этом не можете использовать наследование, потому что оно не дает вам достаточно свободно подбирать части, которые вы хотите использовать.
\end{itemize}

Этот шаблон добавляет сложности в простой процесс создания класса и наполнения его кодом. Каждый концептуальный "объект" становится кластером объектов, который нужно инстанциировать, инициализировать и корректно увязать вместе. Коммуникация между разными компонентами усложняется и размещение в памяти тоже усложняется.

Для большой кодовой базы ее сложность может оправдывать снижение связности и добавление возможности повторного использования кода.


\subsection{Прочие шаблоны программирования}


\subsubsection{Пул объектов}

Пул объектов -- это порождающий шаблон проектирования, который представляет собой набор инициализированных и готовых к использованию объектов. Когда системе требуется объект, он не создаётся, а берётся из пула. Когда объект больше не нужен, он не уничтожается, а возвращается в пул \cite{ProgrammingPatterns}.

Пул объектов обычно применяется для повышения производительности, когда создание объекта в начале работы и уничтожение его в конце приводит к большим затратам. Особенно заметно повышение производительности, когда объекты часто создаются-уничтожаются, но одновременно существует лишь небольшое их число.

При практической реализации шаблона особого внимания заслуживает стратегия поведения в случае переполнения пула. В этом случае возможна одна из трёх стратегий: расширение пула, отказ в создании объекта и аварийный останов, ожидание освобождения одного из объектов в пуле.


\subsubsection{Неизменный объект}

Шаблон Неизменяемый объект налагает запрет на изменение содержимого объекта после того, как объект был создан. Это повышает надежность кода при использовании объекта и уменьшает затраты на параллельный доступ к нему.

При практической реализации шаблона обычно придерживаются следующих правил:
\begin{itemize}
    \item начальной установкой полей объекта занимается конструктор и только он -- не существует методов, которые бы изменяли внутренне состояние объекта;
    \item состояние объекта доступно посредством открытых свойств <<только для чтения>>;
    \item если метод получает информацию о новом состоянии объекта, то в теле метода для сохранения состояния создается новый экземпляр объекта.
\end{itemize}

Можно заметить, что с точки зрения \DotNet, неизменяемые объекты можно рассматривать как попытку смоделировать поведение типов значений при помощи ссылочных типов. Классическим примером неизменяемого объекта в \DotNet является тип \lstinline{string}. При практической реализации неизменяемого объекта могут дополнительно переопределяться операции проверки на равенство и получения копии.


\subsection{Архитектура программного средства}

Проектируемое программное средство кратко можно представить в следующем виде совокупности следующих слабосвязанных компонентов:
\begin{itemize}
    \item игровая логика;
    \item система UI в виде системы экранов, релизованных согласно концепции MVC\footnote{Model-View-Controller};
    \item сетевая система;
    \item внутриигровой магазин и инвентарь;
    \item аудио система.
\end{itemize}

Реализации выше перечисленных модули по большей степени не зависят друг от друга, а контроль над ними осуществляется с помощью специальных классов-<<контроллеров>>. Также контроль над модулями может осуществляется с помощью системы экранов согласно концепции MVC.

Также связь между отдельными модулями и использующими их контроллерами осуществляется с помощью системы событий.