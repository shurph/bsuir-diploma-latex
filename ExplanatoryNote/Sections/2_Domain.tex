\section{Моделирование предметной области}

В разделе сформулированы функциональные требования для проектируемого программного продукта.


\subsection{Описание функциональности программного продукта}


\subsubsection{Игровой процесс}

Создание компьютерных игры начинается как правило с моделирования игрового процесса. В самом общем виде игровой процесс проходит на некотором игровом поле из клеток, на котором два игрока по очереди ставят либо крестик, либо нолик. Тот, кто первый соберёт линию из своих клеток достаточной для выигрыша длины объявляется победителем, но также в процессе игры возможен ничейных исход, когда на поле уже не остаётся места для проведения выигрышной комбинации. Размер поля и количество клеток для выигрыша могут меняются в зависимости от выбранного игрового режима.

В проектируемом программном продукте было решено реализовать по крайней мере два игровых режима:
\begin{itemize}
    \item классический игровой режим, в котором игровой процесс происходит на двухмерном игровом поле размером три на три клетки. Для выигрыша достаточно собрать серию из трёх клеток;
    \item трёхмерный игровой режим, с трёхмерным игровым полем в виде куба с ребром в три клетки и выигрышной комбинацией из трёх клеток.
\end{itemize}

\subsubsection{Компьютерный противник}

Так как разрабатываемая игра требует наличия двух соперников, то для игры без наличия второго игрока требуется наличие компьютерного противника. Компьютерный противник реализуется с помощью различных алгоритмов поиска оптимального хода, которые должны уметь анализировать игровое поле и находить лучшие для себя исходы игры.

В связи с нацеленностью игры на мобильные устройства под управлением операционных систем Android и iOS, которые зачастую ограниченны в ресурсах, требуется достаточно быстрый алгоритм поиска оптимального хода с возможность выбора глубины поиска для обеспечения лучшей производительности на целевых операционных системах.

Также игра должна поддерживать выбор уровня сложности компьютерного противника. Обычно это возможность реализуется заданием вероятности $\text{P}_\text{a}$, с которой противник совершит ход с помощью алгоритма поиска оптимального хода.

Для определения, использовать компьютерному противнику поиск (событие \text{A}) или ходить в случайную клетку (событие $\bar{\text{A}}$) производится имитация случайного события:
\begin{enumerate}[label=\arabic*, itemindent=\parindent + 2.25ex]
    \item С помощью датчика случайных чисел (СЧ) получают СЧ \text{Х}.
    \item Проверяют выполнение неравенства \text{Х} меньше, либо равно $\text{P}_\text{a}$.
    \item Если оно выполняется, то событие \text{A} -- произошло, если нет -- то произошло $\bar{\text{A}}$.
\end{enumerate}


\subsubsection{Многопользовательская игра по сети интернет}

Обеспечение многопользовательской игры по сети интернет является одним из самых сложных этапов в создании компьютерной игры. Необходимый функционал может быть реализован различными способами.

Обычно взаимодействие игровых клиентов строится с помощью архитектуры <<Клиент -- сервер>>. Такая архитектура предполагает наличие выделенного сервера, занимающегося моделированием игрового процесса, и отдельных игроков, которые могут присоединится к серверу. Но как сказано выше, такая архитектура требует наличие выделенного сервера, что не каждый разработчик игры может себе позволить. Для решения этой проблемы можно использовать другую архитектуру под названием <<Клиент -- хост>>.

Архитектура <<Клиент -- хост>> работает по тем же принципам, что и архитектура <<Клиент -- сервер>>, но её главная отличительная черта состоит в том, что один из клиентов выступает так же в качестве сервера. Такая архитектура не требует наличие отдельных серверов моделирования игрового процесса. Игрок может просто создать игровое лобби\footnote{или <<Комната ожидания>> -- место, где собираются игроки перед началом игры} и любой другой игрок сможет присоединится к нему (если конечно количество игроков в лобби не достигло максимума). Но для того, чтобы игрокам был доступен список лобби всё равно требуется отдельный сервер, обеспечивающий координацию игровых сессий.

Игровое лобби представляет собой <<комнату>>, которую игрок может создать, либо присоединится к чужой. Игрок, создавший лобби, является её лидером и имеет возможность исключать других игроков из его. Когда количество игроков в лобби достигло некоторого минимального количество, игрокам позволяется проголосовать за начало игры и если все игроки заявили о своей готовности, то начинается игра.

Что касается игрового процесса, игроки во время игры имеют возможность вернутся обратно в игровое лобби, также после окончания игры у них есть возможность перезапустить игру голосованием без возврата в игровое лобби.

Так как игроки зачастую находятся на больших расстояниях между собой или незнакомы друг другу, то им требуется средство для общения между собой. Для решения этой проблемы можно использовать чат\footnote{\emph{Чат} -- средство обмена сообщениями по компьютерной сети в режиме реального времени.}. С его помощью игрокам можно обмениваться краткими текстовыми сообщения как в игровом лобби, так и во время игры.


\subsubsection{Внутриигровой магазин и инвентарь}

Внутриигровые магазины дают возможность игрокам за деньги покупать в играх различные внутриигровые предметы. Так как происходят операции с реальными деньгами, то внутриигровой магазин должен обеспечивать безопасность транзакций, также магазин должен обеспечить сохранность купленных предметов и проверку на достоверность транзакций чтобы предупредить возможность подделки покупки.

Обычно предметы, продаваемые во внутриигровом магазине, являются косметическими и не влияют на игровой процесс. В разрабатываемой же игре присутствует возможность покупки набора скинов\footnote{\emph{Скин} -- пакет данных, предназначенный для настройки графического интерфейса какой-либо компьютерной программы.} для индикаторов позиции игрока на клетке игрового поля (<<крестика>> или <<нолика>>).

Так как в реализуемой игре магазин предназначен для покупки только единичных копий предметов, то его можно совместить с инвентарём: если предмет не куплен, то будет отображаться кнопка покупки, а если куплен, то отобразится кнопка активации данного предмета.

Также из-за наличия многопользовательской игры требуется синхронизация инвентарей у различных игроков, чтобы во время игрового процесса игрок мог видеть, как свой активированный скин, так и скин противника.


\subsubsection{Пользовательский интерфейс, управление и визуальный стиль}

Чтобы компьютерная игра запомнилась пользователю, требуется наличие в игре удобного пользовательского интерфейса, отзывчивого управления, приятного визуального стиля, различных анимаций и эффектов.

В связи с нацеленностью игры на мобильные платформы преимущественно с сенсорными экранами требуется наличие гибкого и удобного пользовательского интерфейса. В разрабатываемой игре пользовательский интерфейс будет представлять собой систему из отдельных экранов: начальный экран, экран одиночной игры, экран создания лобби, экран настроек, экран магазина и т.п. Экраны же имеют свои наборы элементов управления: различные кнопки, поля ввода, текстовые метки и т.д. В каждый момент времени пользователь наблюдает только один экран, а переходы между экранами могут инициализироваться различными способами: нажатиями кнопок пользователем либо особыми игровыми событиям. Процесс смены экранов же оформляется с помощью анимаций.

Отдельно стоит отметить особенности управления в трёхмерном игровом режиме. Так как игровое поле представлено не в двух, а в трёх измерениях, то требуются и специальные средства для управления игровым процессом. В реализуемом же проекте трёхмерное поле делится на отдельные слои: плоскости XY, YZ, XZ, диагонали, и т.п., а на экране с игровым процессом присутствует элемент управления, позволяющий выбрать отдельный слой для игры.


\subsubsection{Прочая функциональность}

Немаловажную роль в компьютерных играх играет и звуковое сопровождение. В создаваемой же игре аудио система будет представлять собой коллекцию аудио групп со специфичными для каждой настройками (возможность зациклить воспроизведение для группы с музыкой, разрешить нескольким аудио эффектам воспроизводится одновременно для группы со звуковыми эффектами), которые в свою очередь управляют отдельными аудио источниками. Аудио группы и источники обладают некоторым похожим набором возможностей (как например возможность сделать плавное затухание звука и остановка), но с тем различием, что аудио группа делает это операции над всеми своими аудио источниками одновременно.

Также в игре должен быть экран настроек, позволяющий отключать отдельные аудио группы, менять имя игрока, отключать некоторые визуальные эффекты, влияющие на производительность приложения.


\subsection{Спецификация функциональных требований}

На основании анализа исходных данных и функциональных требований для проектируемого программного средства можно составить следующую спецификацию:


\subsubsection{Игровой процесс}

\begin{enumerate}
    \item игровой процесс проходит на некотором игровом поле из клеток, на котором два игрока по очереди ставят либо крестик, либо нолик;
    \item первый игрок, собравший линию из своих клеток достаточной для выигрыша длины объявляется победителем;
    \item в процессе игры возможен ничейных исход, когда на поле уже не остаётся места для проведения выигрышной комбинации;
    \item должны быть реализованны по крайней мере два игровых режима:
    \begin{enumerate}
        \item классический игровой режим, в котором игровой процесс происходит на двухмерном игровом поле размером три на три клетки. Для выигрыша достаточно собрать серию из трёх клеток;
        \item трёхмерный игровой режим, с трёхмерным игровым полем в виде куба с ребром в три клетки и выигрышной комбинацией из трёх клеток.
    \end{enumerate}
\end{enumerate}


\subsubsection{Компьютерный противник}

\begin{enumerate}
    \item должен быть реализован с помощью алгоритма поиска оптимального хода, который должен уметь анализировать игровое поле и находить лучшие для себя исходы игры;
    \item должна быть возможность выбора глубины поиска;
    \item игра должна поддерживать выбор уровня сложности компьютерного противника с помощью задания вероятности, с которой противник совершит ход с помощью алгоритма поиска оптимального хода.
\end{enumerate}


\subsubsection{Многопользовательская игра по сети интернет}

\begin{enumerate}
    \item реализация многопользовательской игры с помощью архитектуры <<Клиент -- хост>>;
    \item должна быть возможность создания игрового лобби:
    \begin{enumerate}
        \item лобби имеет фиксированный размер и не позволяет присоединится игрокам при его достижении;
        \item лидер лобби должен иметь возможность исключать игроков;
        \item возможность игрокам голосовать за начало игры;
        \item во время игры игроки должны иметь возможность вернутся обратно в игровое лобби;
    \end{enumerate}
    \item после окончания игры у игроков должна быть возможность перезапустить игру голосованием без возврата в лобби;
    \item наличие внутриигрового чата.
\end{enumerate}


\subsubsection{Внутриигровой магазин и инвентарь}

\begin{enumerate}
    \item должен обеспечивать безопасность транзакций;
    \item должен обеспечить сохранность купленных предметов;
    \item должен проводить проверку на достоверность транзакций;
    \item внутриигровой магазин должен быть совмещён с инвентарём:
    \begin{enumerate}
        \item если предмет не куплен, то будет отображаться кнопка покупки;
        \item если предмет куплен, то отобразится кнопка активации данного предмета;
    \end{enumerate}
    \item должна происходить синхронизация инвентарей игроков во время многопользовательской игры.
\end{enumerate}


\subsubsection{Пользовательский интерфейс, управление и визуальный стиль}

\begin{enumerate}
    \item пользовательский интерфейс должен представлять собой систему из отдельных экранов;
    \item в каждый момент времени пользователь наблюдает только один экран;
    \item переходы между экранами должны инициализироваться различными способами: нажатиями кнопок пользователем либо особыми игровыми событиям;
    \item процесс смены экранов должен оформляется с помощью анимаций;
    \item в трёхмерном игровом игровое поле должно быть поделено на отдельные слои с возможностью выбора отдельного слоя.
\end{enumerate}


\subsubsection{Прочая функциональность}

\begin{enumerate}
    \item должна быть реализована аудио система в виде коллекции аудио групп со специфичными для каждой настройками;
    \item в игре должен присутствовать экран настроек, со следующими возможностями:
    \begin{enumerate}
        \item возможность отключать отдельные аудио группы;
        \item возможность менять имя игрока;
        \item возможность отключать некоторые визуальные эффекты, влияющие на производительность приложения.
    \end{enumerate}
\end{enumerate}