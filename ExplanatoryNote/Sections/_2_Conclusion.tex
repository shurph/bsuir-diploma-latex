\newpage
\phantomsection
\addcontentsline{toc}{section}{Заключение}
\sectioncentered*{Заключение}

В данном дипломном проекте была поставлена задача разработки компьютерной игры \BinaryWars, представляющей собой продвинутый аналог игры \TicTacToe с множеством дополнительных особенностей: различные игровые режимы (двухмерное игровое поле три на три клетки, трёхмерное игровое поле), возможность играть против компьютерного противника с выбором уровня сложности, кроссплатформенная многопользовательская игра со случайным или определённым оппонентом, наличие внутриигрового магазина.

Так, в ходе выполнения дипломной работы для обеспечения стабильности и расширяемости были рассмотрены различные архитектурные подходы и шаблоны проектирования, многие из которых нашли применение при реализации программного продукта: код программы был разбит на множество компонентов, взаимодействие которых организованно на основе концепции Model-View-Controller и с помощью процесса внедрения зависимостей. Такой подход обеспечивает дополнение и изменение кода без влияния на другие части проекта, а также возможность использовать полученный код в будущих проектах.

Для реализации компьютерного соперника в проекте использовался алгоритм альфа-бета отсечения для поиска оптимального хода, дополненный собственной реализацией эвристической функции, которая позволяет довольно точно определить оценку текущего дерева ходов на игровом поле при меньшей глубине поиска.

Было проделано много работы при создании многопользовательской игры. Для максимальной с точки зрения игрока простоты создания и присоединения к игровым сессиям было решено использовать систему <<матчмейкинга>>, в которой данные о всех игровых сессиях хранятся на удалённом сервере. Игровой клиент обращаются к серверу с целью получения всех доступных игровых сессий и игроку остаётся только выбрать определённую сессию для начала игры, или же игрок может сам создать свою сессию, а остальные пользователи сами смогут к нему присоединиться. Сама же реализация многопользовательской игры построена с использованием архитектуры <<клиент-хост>>, которая предполагает, что один из игроков является одновременно и клиентом, и сервером, что убирает требование наличия выделенного сервера для реализации многопользовательской игры.

Также в ходе работы над дипломным проектом были реализованы другие особенности: аудио система с возможностью группировки отдельных категорий звуковых эффектов и управлением ими как единым целым, внутриигровой магазин с системой инвентаря, система UI, игровой режим на трёхмерном поле.

Разработка компьютерных игр обычно не заканчивается на реализации строго определённого перечня возможностей, а продолжается и дальше после их выпуска. Так, например, в создаваемой в рамках дипломного проекта компьютерной игре, можно реализовать новые игровые режимы или другие виды товаров во внутриигровом магазине. Тем не менее цель дипломного проекта достигнута и уже сейчас полученная игра готова к выпуску на целевых платформах.