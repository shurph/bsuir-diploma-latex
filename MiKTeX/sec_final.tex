\sectioncentered*{Заключение}
\addcontentsline{toc}{section}{Заключение}

В данном дипломном проекте был рассмотрен разработки сетевого менеджера паролей с шифрованием/дешифрованием данных посредством системы криптографии с открытым ключом.
В рамках дипломного проекта была разработана клиент-серверное приложение для представления и защиты конфиденциальных пользовательских данных.
В разработанном приложении использовались два различных подхода к оценке стойкости генерируемых ключей, на основе принципа функции Эйлера и оценке апостериорной вероятности.
Также для оптимизации вычислительных методов использовались стратегия построения пространства каскадов методов, для решения всех возможных решений.

В результате было получено кроссплатформенное приложение, удовлетворяющее исходным задачам, которые ставились задание на дипломное проектирование.
Результаты работы реализованных в качестве исполняемого модуля в большинстве случаях превосходят по качеству функциональность уже существующего программного обеспечения.
Также был предложен способ улучшения качества поиска простых чисел на малом объеме данных, основанный на предварительной рандомизации экспериментальных данных.
Данный способ удовлетворительно зарекомендовал себя в проведённых тестах.
Помимо предложенной модификации были произведены небольшие улучшения в хорошо известных алгоритмах, направленные на повышение скорости их работы.
Для повышения производительности применялась мемоизация и использовались прологарифмированные версии некоторых методов.

%В итоге получилось раскрыть тему дипломного проекта и создать в его рамках программное обеспечение.
В результате цель дипломного проекта была достигнута.
Было создано программное обеспечение.
Но за рамками рассматриваемой темы осталось еще много других алгоритмов шифрования и интересных вопросов, связанных, например, со статистическим выводом суждений в стойкости синхротронных алгоритмов, нахождением простых чисел для генерации устойчивых ключей и других вопросов, возникающих при работе над приложением.
Эти задачи также являются нетривиальными и требуют детального изучения и проработки.
В дальнейшем планируется развивать и довести существующее программное обеспечение до полноценной сетевого менеджера, способного управлять конфиденциальными данными различной сложности.