\section{Тестирование программного средства}
\label{testing}

Тестирование "--- процесс анализа или эксплуатации программного обеспечения с целью выявления в нем дефектов, ошибок~\cite{testing_bahtizin}.
Тестирование является важным процессом при разработке программных средств.
Оно позволяет выявить дефекты и ошибки на ранних стадиях разработки и повысить качество программного обеспечения.

Одним из видов тестирования является модульное тестирование.
При данном виде тестирования его объектом является отдельный модуль разрабатываемого программного средства.
В случае, если данному модулю для его корректной работы необходимы некоторые другие модули, вокруг него выстраиваются так называемые
<<строительные леса>>, которые эмулируют поведение других модулей и позволяют протестировать модуль-объект в изоляции.

Для многих языков программирования существуют готовые решения для автоматизированного модульного тестирования кода.
Данный дипломный проект разрабатывался на трех языках программирования: Ruby, Rust и С.

Для Ruby существует несколько готовых решений по модульному тестированию, таких как rspec и minitest.
В данном дипломном проекте для модульного тестирования кода использовался minitest, по причине его простоты по сравнению с rspec.

Для языка программирования Rust существует встроенная в сам язык решение по модульному тестированию.
Данное решение использует метаатрибуты, которыми могут помечаться функции в Rust.
Для использования данного решения достаточно помечать тестовые функции метаатрибутом test, и внутри них делать проверки на соответствие ожидаемым результатам с помощью макроса assert.

Для языка программирования С существует множество готовых решения для модульного тестирования: cunit, cfix, MICRO\_UNIT и другие.
Однако в данном дипломном проекте на языке программирования С написана лишь небольшая и простая часть (модуль отображения результатов),
поэтому от модульного тестирования было решено отказаться.

Также в целях проверки соответствия разрабатываемого приложения заявленным функциональным требованиям, был разработан ряд вариантов тестирования.
Предложенные варианты тестирования представлены в таблице~\ref{testing:test_cases}.

\begin{longtable}{| >{\centering}m{0.05\textwidth} 
                  | >{\centering}m{0.15\textwidth} 
                  | >{\centering}m{0.20\textwidth} 
                  | >{\centering}m{0.20\textwidth} 
                  | >{\centering\arraybackslash}m{0.25\textwidth}|}
\caption{Варианты тестирования} \label{testing:test_cases} \tabularnewline
  \hline Но\-мер & На\-и\-ме\-но\-ва\-ние & Входные данные & Инструкция для получения результата & Ожидаемый результат \tabularnewline
  \hline 1 & Вос\-про\-из\-ве\-де\-ние ранее записанной симуляции & Записанный результат симуляции &
             Запустить модуль отображения результатов, подав на вход записанный ранее результат симуляции &
             Модуль отображения результатов корректно воспроизводит результат симуляции \tabularnewline
  \hline 2 & Симуляция в режиме реального времени & Сценарий симуляции &
             Запустить симуляцию в режиме реального времени &
             Результат симуляции успешно отображается с помощью модуля отображения результатов в режиме реального времени \tabularnewline
  \hline 3 & Симуляция в режиме реального времени с одной сотней пешеходов & Сценарий симуляции, обеспечивающий наличие одновременно одной сотни пешеходов в симуляции &
             Запустить симуляцию в режиме реального времени &
             Результат симуляции успешно отображается с помощью модуля отображения результатов в режиме реального времени \tabularnewline
  \hline
\end{longtable}
