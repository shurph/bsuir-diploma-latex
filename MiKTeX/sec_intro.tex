\sectioncentered*{Введение}
\addcontentsline{toc}{section}{Введение}
\label{sec:intro}

Задачa данного курсового проекта - разработка микро-ЭВМ, которая обладает определенными особенностями в соответствии с заданным вариантом.

Одной из ключевых особенностей разрабатываемой микро-ЭВМ является принстонский тип архитектуры. Основной отличительной особенностью такой архитектуры является хранение данных и команд микро-ЭВМ в одном адресном пространстве.
Главный недостаток такой схемы - ограничение пропускной способности между процессором и памятью по сравнению с объёмом памяти.
Но с другой стороны она обладает плюсами: можно производить над командами те же операции, что и над числами, и, соответственно, открывается ряд возможностей.

Разрабатываемая микро-ЭВМ обладает двумя модулями памяти размером в 1024 слов по 16 байт, то есть общий объём памяти составляет 32 Кбайт (16 кбайт данных и 16 кбайт для хранения команд).

Разрабатываемая архитектура представляет собой архитектуру \- 16-разрядного процессора с дополнительной внутренней шиной данных. Это позволяет работать с относительно большими числами за один такт процессорного времени.

В данной архитектуре были реализованы два вида адресации: косвенная регистровая и прямая.
Косвенная регистровая адресация позволяет задавать в поле команды адрес регистра, хранящего адрес ячейки памяти, в которой находится операнд.
Это позволяет работать с памятью через указатели - можно менять содержимое регистра, не затрагивая основную память.
Такой способ адресации позовляет эффективно работать с массивами данных.
Прямая адресация позволяет задавать ячейку памяти или номер регистра, с которым будем работать. Недостатком такой операции является, например, невозможность адресовать массивы данных поэлементно, преимущество - обращение к данным происходит быстро.

Из дополнительных особенностей стоит упомянуть о наличии десяти регистров общего назначения, стека размером 11 слов, арифметико-логического устройства, выполняющего 4 операции.
