\sectioncentered*{Введение}
\addcontentsline{toc}{section}{Введение}
\label{sec:intro}

На момент разработки дипломного проекта, как сообщает американская компания по организации информационной безопасности Hold Security~\cite{cyber_vor}, с начала текущего года лишь одной группой хакеров было похищено более двух миллионов различных пользовательских данных.

Несмотря на то, что существует программное обеспечение, которое способно решить эту проблему, пользователи предпочитают хранить свои данные на общедоступных ресурсах, подвергая их риску быть похищенным. Особенностью разрабатываемого приложения является криптосистема, отвечающая современным стандартам~\cite{ecma_335}, и использование собственных алгоритмов оптимизации скорости вычисления шифров.

Использование менеджеров паролей является обязательным условием для эффективного управления и хранения пользовательских данных. Это обусловлено единством интерфейсов и полным сокрытием реальных данных, посредством шифрования.

Вопрос коммерциализации подобного програмного обеспечения находится в стадии становления. Главной задачей его участников является развитие клиентской базы и превращение технологических идей в прибыльный бизнес. При этом необходимо решить множество организационных, технологических и финансовых вопросов.

Все продукты, направленные на управление данными пользователя, различаются по многим параметрам: временным и трудовым затратам, сложностью, требуемой надежности системы и так далее. Эти параметры влияют на стоимость разработки и сложность взаимодействия. Для успешной реализации любого крупного проекта недостаточно только выбрать эффективную технологию и средство разработки.

Главной задачей, при разработки дипломного проекта, ставился вопрос оптимизации существующих алгоритмов шифрования и оценки криптографической стойкости системы; создания кроссплатформной архитектуры, с полной поддержкой всего функционала в различных средах использования, включая любую общедоступную сеть.

Целью дипломного проекта является реализация сетевого менеджера паролей с шифрованием данных. Задачей проекта ставится исправление известных недостатков существующих модулей криптографии, применяемых в сетевых системах со сложным интерфейсом администрирования, создание простого, наглядного интерфейса, а так же примесь новых средств для управления данными.