\chapter*{ОБЩАЯ ХАРАКТЕРИСТИКА РАБОТЫ}
\addcontentsline{toc}{chapter}{ОБЩАЯ ХАРАКТЕРИСТИКА РАБОТЫ}

\label{sec:general_characteristics}

\subsection*{\textbf{Цель и задачи исследования}}

Цель магистерской диссертации "--- провести исследование влияния массовой паники на поведение людей в общем
и на характеристики пешеходных потоков в частности, а также оценить степень и механизмы влияния массовой паники
на время эвакуации.

Для достижения поставленной цели необходимо решить следующие задачи:

\begin{itemize}
  \item Провести исследование существующих моделей массовой паники: выделить основные атрибуты моделей, оценить их результаты.
  \item На основе исследования существующих моделей разработать собственную модель, учитывающую некоторые дополнительные аспекты рассматриваемой области.
  \item Разработать программное средство, использующее модель массовой паники собственной разработки.
  \item Провести эксперименты, позволяющие оценить влияние массовой паники на исследуемые характеристики.
\end{itemize}

Объектом магистерской диссертации является явление массовая паники.

Предметом магистерской диссертации является влияние массовой паники на характеристики пешеходных потоков и на время эвакуации.
Основная гипотеза, положенная в основу работы: массовая паника негативно сказывается на времени эвакуации.

\subsection*{\textbf{Личный вклад соискателя}}

Результаты, приведенные в диссертации, получены соискателем лично.
Вклад научного руководителя \mastersCharSupervisor, заключается в формулировке цели и задач исследования.

\subsection*{\textbf{Опубликованность результатов диссертации}}

По теме диссертации опубликована 1 печатная работа в сборниках трудов и материалов международных конференций.

\subsection*{\textbf{Структура и объем диссертации}}

Диссертация состоит из введения, общей характеристики работы, пяти глав, заключения, библиографического списка и одного приложения.
В главе 1 приводится обзор существующей научной литературы на обсуждаемую тему. Глава 2 посвящена описанию разработанной модели массовой паники.
Глава 3 включает в себя описание модификаций, выполненных на программном средстве, позволяющих ему использовать разработанную в главе 2 модель.
В главе 4 приводится краткое измененное руководство пользователя к разработанному программному средству.
Глава 5 описывает проведенные эксперименты и анализирует их результаты.

Общий объем работы составляет 73 страницы, из которых основного текста "--- 73 страницы, 17 рисунков на 13 страницах, 1 листинг исходных кодов на 20 страницах,
список использованных источников из 32 наименований на 3 страницах.

