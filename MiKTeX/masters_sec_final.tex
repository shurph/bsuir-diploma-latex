\chapter*{ЗАКЛЮЧЕНИЕ}
\addcontentsline{toc}{chapter}{ЗАКЛЮЧЕНИЕ}

\label{sec:final}

В рамках данной работы была разработана модель массовой паники на основе набора предложенных в научной литературе моделей
с некоторыми уникальными улучшениями.
Было также программное средство по расчету времени эвакуации с использованием разработанной модели на основе существующего
программного средства определения вероятных мест скопления людей в помещениях.

Был проведен ряд экспериментов для исследования влияния компонентов модели на время эвакуации.

Выяснилось, что некоторые компоненты модели не оказывают значительного влияния на исследуемую характеристику.
К данным компонентам можно отнести силу <<стадного поведения>> и силу физического контакта.
Причины, по которым возникла данная ситуация, разные для этих двух сил.
В случае силы <<стадного поведения>> причиной служат упрощения в выборе маршрута движения,
сделанные при разработке программного средства.
В случае силы физического контакта причиной является неполная модель движения пешеходов в общем,
и отсутствие инерции и обработки столкновений в частности.
Хотя данный результат является негативным, он указывает на определенные недостатки в модели и программном
средстве, что позволит в будущем их улучшить.

Другие компоненты модели показали лучшие результаты.
Сила случайных флуктуаций, представляющая случайные иррациональные действия пешехода,
оказывает сильно негативное воздействие на время эвакуации.
Механизм данного воздействия достаточно прост "--- сила случайных флуктуаций мешает
пешеходам двигаться к своей цели (уводит их в сторону), что приводит к более длинному пути
к цели и к более длительной эвакуации. При превышении определенного порога мощности
сила случайных флуктуаций вызывает возмущения в потоках людей, что приводит к еще большему
увеличению времени эвакуации.

Модель распространения паники также можно отнести к удачным компонентам.
К сожалению, не было найдено объективных характеристик по которым можно было бы оценить данную модель.
Однако визуально модель распространения паники показывает хорошие результаты.

При работе с разработанным программным средством также были выявлены некоторые недостатки.
Основным недостатком можно назвать отсутствие инструментария для генерации схемы сооружения.
Схема сооружения "--- файл формата SVG "--- пишется в текстовом редакторе, что занимает значительное количество времени.
Например, схема сооружения офиса используемая в исследовании была создана с помощью набора скриптов по генерации элементов
на языке программирования Ruby, и даже в таком случае ее создание заняло около 36 часов чистого времени.
Также к недостаткам программного средства можно отнести отсутствие интерактивного режима воспроизведения,
при котором можно было бы приостановить симуляцию для выполнения каких-либо действий, контролировать скорость
воспроизведения симуляции и др.

В целом, разработанный комплекс модели паники и программного средства представляет интерес для исследования,
однако пока еще не готов для применения на реальных сооружениях.
В дальнейшем работа над данным комплексом должна вестись в направлениях:

\begin{itemize}
  \item разработки редактора схем сооружения;
  \item повышения стабильности и воспроизводимости результатов;
  \item разработки интерактивного режима воспрозведения симуляции;
  \item разработки модели поиска маршрута для каждого пешехода;
  \item разработки модели движения с инерцией и столкновениями.
\end{itemize}

Каждый из этих пунктов требует значительных трудозатрат, однако при условии их выполнения программное средство потенциально
может стать очень полезным инструментом при оценке времени эвакуации из любого сооружения.
