\newcommand{\byr}{ тыс. бел. руб}

\section{Технико-экономическое обоснование эффективности разработки и использования сетевого менеджера паролей с шифрованием/дешифрованием данных}
\label{sec:econ}

\subsection{Характеристика программного продукта}
\label{sub:econ:desc}

Большое значение для работодателя в современном мире имеет планирование и управление ресурсами и процессами на целом предприятии.
«Сетевой менеджер паролей с шифрованием/дешифрованием данных» представляет собой систему управления пользовательскими данными, имеющими конфиденциальный характер.

Подобные системы внедряются для того, чтобы объединить все данные и все необходимые функции в одной компьютерной системе, которая будет обслуживать текущие потребности пользователей в защите конфиденциальных данных.

Система ведет единую базу данных по всем записям и их группам, так что доступ к информации становится проще, а главное, пользователь получает возможность получить доступ к ней из-под любой платформы.

Экономическая целесообразность инвестиций в разработку и использование программного продукта осуществляется на основе расчета и оценки следующих показателей:
\begin{itemize}
  \item чистая дисконтированная стоимость $ \text{ЧДД} $;
  \item срок окупаемости инвестиций $ \text{ТОК} $;
  \item рентабельность инвестиций $ \text{Р}_\text{и} $.
\end{itemize}

Разработка проектов программных средств связана со значительными затратами ресурсов (трудовых, материальных, финансовых). В связи с этим создание и реализация каждого проекта программного обеспечения нуждается в соответствующем технико-экономическом обосновании (ТЭО).
Для оценки экономической эффективности инвестиционного проекта по разработке и внедрению программного продукта необходимо рассчитать:
\begin{itemize}
  \item результат $ \text{Р} $, получаемый от использования программного продукта;
  \item затраты (инвестиции), необходимые для разработки программного продукта;
  \item показатели эффективности инвестиционного проекта по производству приложения «Сетевой менеджер паролей с шифрованием/дешифрованием данных».
\end{itemize}

\subsection{Расчет стоимостной оценки затрат}
\label{sub:econ:calc_total}

Общие капитальные вложения $ \text{К}_\text{о} $ заказчика (потребителя), связанные с приобретением, внедрением и использованием ПС, рассчитываются по формуле~(\ref{eq:econ:calc_total:totalProgramSize}):
\begin{equation}
  \label{eq:econ:calc_total:totalProgramSize}
  \text{K}_\text{o} = \text{К}_\text{пр} + \text{К}_\text{ос} \text{\,,}
\end{equation}
\begin{explanation}
где & $ \text{К}_\text{пр} $ & затраты пользователя на приобретение ПС по отпускной цене разработчика с учетом стоимости услуг по эксплуатации и сопровождению (тыс. руб.); \\
    & $ \text{К}_\text{ос} $ & затраты пользователя на освоение ПС (тыс. руб.).
\end{explanation}

\subsection{Расчет затрат на разработку и отпускной цены программного продукта}
\label{sub:econ:calc_total_develop}

Основная заработная плата~\cite[с.~59,~приложение 1]{palicyn_2006} исполнителей на разработку приложения «Сетевой менеджер паролей с шифрованием/дешифрованием данных» рассчитывается по формуле~(\ref{eq:econ:calc_total_develop:total_program_size}):
\begin{equation}
  \label{eq:econ:calc_total_develop:total_program_size}
  \text{З}_\text{o} = \sum_{i = 1}^{n}
    \text{Т}_\text{чi} \cdot
    \text{K}\cdot
    \text{Ф}_\text{n}
    \text{\,,}
\end{equation}
\begin{explanation}
где & $ \text{T}_{чi} $ & часовая тарифная ставка i-го исполнителя (тыс. руб.); \\
    & $ \text{Ф}_{n} $ & плановый фонд рабочего времени i-го исполнителя (дн.); \\
    & $ \text{T}_{ч} $ & количество часов работы в день (ч); \\
    & $ \text{K} $ & коэффициент премирования; \\
    & $ n $ & количество исполнителей, занятых разработкой приложения.
\end{explanation}

Коэффициент премирования 1,5. Для расчета заработной платы месячная тарифная ставка 1-го разряда на предприятии установлено на уровне 1850 \byr.
Данные расчетов сведены в таблицу~(\ref{table:econ:calc_total_develop:function_sizes}).

\begin{table}[ht]
\caption{Расчет заработной платы}
\label{table:econ:calc_total_develop:function_sizes}
\centering
  \begin{tabular}{| >{\raggedright}m{0.18\textwidth}
                  | >{\centering}m{0.10\textwidth}
                  | >{\centering}m{0.15\textwidth}
                  | >{\centering}m{0.15\textwidth}
                  | >{\centering}m{0.12\textwidth}
                  | >{\centering\arraybackslash}m{0.14\textwidth}|}
   \hline
   Категория исполнителя & Разряд & Тарифный коэф"=фициент & Часовая тарифная ставка, тыс. руб. & Трудоем"=кость, дн. & Основная заработная плата, тыс. руб. \\
   \hline
   Программист \Rmnum{2}-категории & $ \num{12} $ & $ \num{2,84} $ & $ \num{29,852} $ & $ \num{32} $ & $ \num{11463,273} $\\
   \hline
   Ведущий программист & $ \num{15} $ & $ \num{3,48} $ & $ \num{36,579} $ & $ \num{32} $ & $ \num{14046,546} $ \\
   \hline
   Начальник, руководитель проекта & $ \num{16} $ & $ \num{3,72} $ & $ \num{39,102} $ & $ \num{16} $ & $ \num{7507,636} $\\
   \hline
   Итого с премией (50\%), $ \text{З}_\text{о} $ & $ - $ & $ - $ & $ - $ & $ - $ & $ \num{33017,455} $\\
   \hline
  \end{tabular}
\end{table}

Дополнительная заработная плата на наш программный продукт $ \text{З}_{\text{д}} $, включает выплаты, предусмотренные законодательством о труде (оплата отпусков, льготных часов, времени выполнения государственных обязанностей и других выплат), и определяется по нормативу в процентах к основной заработной плате по формуле~(\ref{eq:econ:calc_total_develop:additional_salary}):
\begin{equation}
  \label{eq:econ:calc_total_develop:additional_salary}
  \text{З}_{\text{д}} =
    \frac {\text{З}_{\text{о}} \cdot \text{Н}_{\text{д}}}
          {100\%} \text{\,,}
\end{equation}
\begin{explanation}
  где & $ \text{Н}_{\text{д}} $ & норматив дополнительной заработной платы, $ \% $.
\end{explanation}

Приняв норматив дополнительной заработной платы $ \text{Н}_{\text{д}} = \num{10\%} $ и подставив известные данные в формулу~(\ref{eq:econ:calc_total_develop:additional_salary}) получим
\begin{equation}
  \label{eq:econ:calc_total_develop:additional_salary_calc}
  \text{З}_{\text{д}} =
    \frac{\num{33017,455} \times 10\%}
         {100\%} \approx \SI{3301,745}{\text{\byr}} \text{\,.}
\end{equation}

Отчисления в фонд социальной защиты населения и обязательное страхование $ \text{З}_{сз} $ определяются в соответствии с действующими законодательными актами по нормативу в процентном отношении к фонду основной и дополнительной зарплаты исполнителей, определенной по нормативу, установленному в целом по организации. Общие отчисления на социальную защиту рассчитываются по формуле~(\ref{eq:econ:calc_total_develop:additional_salary}):
\begin{equation}
  \label{eq:econ:calc_total_develop:soc_prot}
  \text{З}_{\text{сз}} =
    \frac{(\text{З}_{\text{о}} + \text{З}_{\text{д}}) \cdot \text{Н}_{\text{сз}}}
         {\num{100\%}} \text{\,,}
\end{equation}
\begin{explanation}
  где & $ \text{Н}_{\text{сз}} $ & норматив отчислений в фонд социальной защиты населения и на обязательное страхование.
\end{explanation}

Подставив вычисленные ранее значения в формулу~(\ref{eq:econ:calc_total_develop:soc_prot}) получаем:
\begin{equation}
  \label{eq:econ:calc_total_develop:soc_prot_calc}
  \text{З}_{\text{сз}} =
    \frac{ (\num{3017,455} + \num{3301,745}) \times \num{34,6\%} }
         { \num{100\%} }
    \approx \SI{12566,443}{\text{\byr}} \text{\,.}
\end{equation}

Расходы по статье «Машинное время» $ \text{Р}_{м} $ включают оплату машинного времени, необходимого для разработки и отладки программного продукта~\cite[с.\,69, приложениe~6]{palicyn_2006}, которое определяется по нормативам (в машино-часах) на \num{100} строк исходного кода $ \text{H}_{мв} $ машинного времени, и определяются по формуле~(\ref{eq:econ:calc_total_develop:soc_prot}):
\begin{equation}
  \label{eq:econ:calc_total_develop:machine_time}
  \text{Р}_{\text{м}} =
    \text{Ц}_{\text{м}} \cdot
    \text{Т}_{\text{пр}} \text{\,,}
\end{equation}
\begin{explanation}
  где & $ \text{Ц}_{\text{м}} $ & цена одного машино-часа. Рыночная стоимость машино-часа компьютера со всеми необходимым оборудованием, \byr; \\
      & $ \text{T}_{\text{пр}} $ & время работы над программным продуктом, ч.
\end{explanation}

Цена одного часа машинного времени составляет $ \text{Ц}_{\text{м}} = \SI{18}{\text{\byr}} $.
Общее время, затраченное на разработку программного продукта равно \num{252} часа.
Подставляя известные данные в формулу~(\ref{eq:econ:calc_total_develop:machine_time}) получаем:
\begin{equation}
  \label{eq:econ:calc_total_develop:machine_time_calc}
  \text{Р}_{\text{м}} =
    \num{18} \cdot
    \num{252} =
    \SI{4536}{\text{\byr}} \text{\,.}
\end{equation}

Расходы по статье «Научные командировки» $ \text{Р}_{нк} $ на программное средство определяются по формуле~(\ref{eq:econ:calc_total_develop:soc_prot}):
\begin{equation}
  \label{eq:econ:calc_total_develop:business_trip}
  \text{Р}_{\text{нк}} =
    \frac{ \text{З}_{\text{о}} \cdot \text{Н}_{\text{рнк}} }
         { \num{100\%} } \text{\,,}
\end{equation}
\begin{explanation}
  где & $ \text{Н}_{\text{рнк}} $ & норматив расходов на командировки в целом по организации,~$ \% $.
\end{explanation}

Подставляя ранее вычисленные значения в формулу~(\ref{eq:econ:calc_total_develop:business_trip}) и приняв значение $ \text{Н}_{\text{к}} = \num{10\%} $ получаем:
\begin{equation}
  \label{eq:econ:calc_total_develop:business_trip_calc}
    \text{Р}_{\text{нк}} =
    \frac{ \num{33017,455} \times \num{10\%} }
         { \num{100\%} } = \SI{3301,745}{\text{\byr}} \text{\,.}
\end{equation}

Расходы по статье «Прочие затраты» $ \text{П}_{з} $ на программное средство включают затраты на приобретение и подготовку специальной научно-технической информации и специальной литературы. И определяются по формуле~(\ref{eq:econ:calc_total_develop:other_cost}):
\begin{equation}
  \label{eq:econ:calc_total_develop:other_cost}
  \text{П}_{\text{з}} =
    \frac{ \text{З}_{\text{о}} \cdot \text{Н}_{\text{пз}} }
         { \num{100\%} } \text{\,,}
\end{equation}
\begin{explanation}
  где & $ \text{Н}_{\text{пз}} $ & норматив прочих затрат в целом по организации,~$ \% $.
\end{explanation}

Приняв значение норматива прочих затрат $ \text{Н}_{\text{пз}} = \num{20\%} $ и подставив вычисленные ранее значения в формулу~(\ref{eq:econ:calc_total_develop:other_cost}) получаем:
\begin{equation}
  \label{eq:econ:calc_total_develop:other_cost_calc}
  \text{П}_{\text{з}} =
    \frac{ \num{33017,455} \times \num{20\%} }
         { \num{100\%} } =
    \SI{6603,491}{\text{\byr}} \text{\,.}
\end{equation}

Затраты по статье «Накладные расходы» $ \text{Р}_{\text{н}} $, связанные с необходимостью содержания аппарата управления, вспомогательных хозяйств и опытных (экспериментальных) производств, а также с расходами на общехозяйственные нужды $ \text{Р}_{\text{н}} $, и определяют по формуле~(\ref{eq:econ:calc_total_develop:other_cost}):
\begin{equation}
  \label{eq:econ:calc_total_develop:overhead_cost}
  \text{Р}_{\text{н}} =
    \frac{ \text{З}_{\text{о}} \cdot \text{Н}_{\text{рн}} }
         { \num{100\%} } \text{\,,}
\end{equation}
\begin{explanation}
  где & $ \text{Н}_{\text{рн}} $ & норматив накладных расходов в организации,~$ \% $.
\end{explanation}

Приняв норму накладных расходов $ \text{Н}_{\text{рн}} = \num{100\%} $ и подставив известные данные в формулу~(\ref{eq:econ:calc_total_develop:overhead_cost}) получаем:
\begin{equation}
  \label{eq:econ:calc_total_develop:overhead_cost_calc}
  \text{Р}_{\text{н}} =
    \frac{ \num{33017,455} \times \num{100\%} }
         { \num{100\%} } =
    \SI{33017,455}{\text{\byr}} \text{\,.}
\end{equation}

Общая сумма расходов по смете  $ \text{С}_{\text{р}} $ на программный продукт рассчитывается по формуле~(\ref{eq:econ:calc_total_develop:overhead_cost}):
\begin{equation}
  \label{eq:econ:calc_total_develop:overall_cost}
  \text{С}_{\text{р}} =
    \text{З}_{\text{о}} +
    \text{З}_{\text{д}} +
    \text{З}_{\text{сз}} +
    %\text{Н}_{\text{е}} +
    \text{М} +
    % \text{Р}_{\text{с}} + % спецоборудование не нужно
    \text{Р}_{\text{м}} +
    \text{Р}_{\text{нк}} +
    \text{П}_{\text{з}} +
    \text{Р}_{\text{н}}\text{\,.}
\end{equation}

Подставляя ранее вычисленные значения в формулу~(\ref{eq:econ:calc_total_develop:overall_cost}) получаем:
\begin{equation}
  \label{eq:econ:calc_total_develop:overall_cost_calc}
  \text{С}_{\text{р}} = \SI{96344,344}{\text{\byr}} \text{\,.}
\end{equation}

Расходы на сопровождение и адаптацию, которые несет производитель ПО, вычисляются по нормативу от суммы расходов по смете и рассчитываются по формуле~(\ref{eq:econ:calc_total_develop:software_support}):
\begin{equation}
  \label{eq:econ:calc_total_develop:software_support}
  \text{Р}_{\text{са}} =
    \frac { \text{С}_{\text{р}} \cdot \text{Н}_{\text{рса}} }
          { \num{100\%} } \text{\,,}
\end{equation}
\begin{explanation}
  где & $ \text{Н}_{\text{рса}} $ & норматив расходов на сопровождение и адаптацию ПО,~$ \% $.
\end{explanation}

Приняв значение норматива расходов на сопровождение и адаптацию $ \text{Н}_{\text{рса}} = \num{10\%} $ и подставив ранее вычисленные значения в формулу~(\ref{eq:econ:calc_total_develop:software_support}) получаем:
\begin{equation}
  \label{eq:econ:calc_total_develop:software_support_calc}
  \text{Р}_{\text{са}} =
    \frac { \num{96344,334} \times \num{10\%} }
          { \num{100\%} } \approx \SI{9634,433}{\text{\byr}} \text{\,.}
\end{equation}

Общая сумма расходов на разработку (с затратами на сопровождение и адаптацию) как полная себестоимость программно продукта $ \text{С}_{\text{п}} $ определяется по формуле~(\ref{eq:econ:calc_total_develop:base_cost}):
\begin{equation}
  \label{eq:econ:calc_total_develop:base_cost}
  \text{С}_{\text{п}} = \text{С}_{\text{р}} + \text{Р}_{\text{са}} \text{\,.}
\end{equation}

Подставляя известные значения в формулу~(\ref{eq:econ:calc_total_develop:base_cost}) получаем:
\begin{equation}
  \label{eq:econ:calc_total_develop:base_cost_calc}
  \text{С}_{\text{п}} = \num{96344,334} + \num{9634,433} = \SI{105978,768}{\text{\byr}} \text{\,.}
\end{equation}

Разрабатываемое ПО является заказным, т.\,е.~разрабатывается для одного заказчика на заказ.
На основании анализа рыночных условий и договоренности с заказчиком об отпускной цене прогнозируемая рентабельность проекта составит~$ \text{У}_{\text{рп}} = \num{25\%} $.
Прибыль рассчитывается по формуле~(\ref{eq:econ:calc_total_develop:income}):
\begin{equation}
  \label{eq:econ:calc_total_develop:income}
  \text{П}_{\text{о}} =
    \frac { \text{С}_{\text{п}} \cdot \text{У}_{\text{рп}} }
          { \num{100\%} } \text{\,,}
\end{equation}
\begin{explanation}
  где & $ \text{П}_{\text{о}} $ & прибыль от реализации ПО заказчику, \byr; \\
      & $ \text{У}_{\text{рп}} $ & уровень рентабельности ПО,~$ \% $; \\
      & $ \text{C}_{\text{п}} $ & себестоимость программного продукта,~\byr.
\end{explanation}

Подставив известные данные в формулу~(\ref{eq:econ:calc_total_develop:income}) получаем прогнозируемую прибыль от реализации ПО:
\begin{equation}
  \label{eq:econ:income_calc}
  \text{П}_{\text{с}} =
    \frac { \num{105978,768} \times \num{25\%} }
          { \num{100\%} }
    \approx \SI{264694,692}{\text{\byr}} \text{\,.}
\end{equation}

Прогнозируемая цена нашего программного продукта без налогов, вычисляется по формуле~(\ref{eq:econ:calc_total_develop:estimated_price}):
\begin{equation}
  \label{eq:econ:calc_total_develop:estimated_price}
  \text{Ц}_{\text{п}} = \text{С}_{\text{п}} + \text{П}_{\text{о}}  \text{\,.}
\end{equation}

Подставив данные в формулу~(\ref{eq:econ:calc_total_develop:estimated_price}) получаем цену ПО без налогов
\begin{equation}
  \label{eq:econ:calc_total_develop:estimated_price_calc}
  \text{Ц}_{\text{п}} = \num{96344,344}  + \num{26494,692} = \SI{122839,026}{\text{\byr}} \text{\,.}
\end{equation}

\subsection{Расчет стоимостной оценки результата}
\label{sub:econ:calc_price_total}

Результатом $ \text{Р} $ в сфере использования нашего программного продукта является прирост чистой прибыли и амортизационных отчислений~\cite[с.~166\,--\,167]{crundwell_2008}.

\subsubsection{Расчет прироста чистой прибыли}
\label{sub:econ:calc_price_total:free_money}

~, который представляет собой экономию затрат на заработную плату и начислений на заработную плату, полученную в результате внедрения программного продукта, вычисляется по формуле~(\ref{eq:econ:calc_price_total:free_money:free}):
\begin{equation}
  \label{eq:econ:calc_price_total:free_money:free}
  \text{Э}_{\text{з}} =
          { \text{К}_{\text{пр}} \cdot
            \text{(}
            \text{t}_{\text{c}} \cdot
            \text{Т}_{\text{с}} -
            \text{t}_{\text{н}} \cdot
            \text{Т}_{\text{н}}
            \text{)} \cdot
            \text{N}_{\text{n}} \cdot
            \text{(}\num{1} + \frac {\text{H}_{\text{дn}}} {\num{100\%}}\text{)} \cdot
            \text{(}\num{1} + \frac {\text{H}_{\text{нno}}} {\num{100\%}}\text{)}
          } \text{\,,}
\end{equation}
\begin{explanation}
  где & $ \text{N}_{\text{n}} $ & плановый объем работ по анализу и обработки результатов, сколько раз выполнялись в году; \\
      & $ \text{t}_{\text{c}} $ & трудоемкость выполнения работы до внедрения программного продукта, нормо.~ч; \\
      & $ \text{t}_{\text{n}} $ & трудоемкость выполнения работы после внедрения программного продукта, нормо.~ч; \\
      & $ \text{T}_{\text{c}} $ & часовая тарифная ставка, соответствующая разряду выполняемых работ до внедрения программного продукта,~\byr;\\
      & $ \text{T}_{\text{n}} $ & часовая тарифная ставка, соответствующая разряду выполняемых работ после внедрения программного продукта,~\byr; \\
      & $ \text{К}_{\text{пр}} $ & коэффициент премий,~\byr; \\
      & $ \text{Н}_{\text{д}} $ & норматив дополнительной заработной платы,~$ \% $; \\
      & $ \text{Н}_{\text{по}} $ & ставка отчислений в ФСЗН и обязательное страхование,~$ \% $.
\end{explanation}

Приняв значение планового объема работ по анализу и обработки результатов $ \text{N}_{\text{n}} = \num{11} $, трудоемкость выполнения работы до внедрения программного продукта $ \text{t}_{\text{c}} = \num{170} $, трудоемкость выполнения работы после внедрения программного продукта $ \text{t}_{\text{n}} = \num{24} $, часовая тарифная ставка, соответствующая разряду выполняемых работ до внедрения программного продукта $ \text{T}_{\text{c}} = \num{12} $~\byr, часовая тарифная ставка, соответствующая разряду выполняемых работ после внедрения программного продукта $ \text{T}_{\text{n}} = \num{12} $~\byr, коэффициент премий $ \text{К}_{\text{пр}} = \num{1,5} $~\byr, норматив дополнительной заработной платы $ \text{Н}_{\text{д}} = \num{20\%} $, ставку отчислений в ФСЗН и обязательное страхование $ \text{Н}_{\text{по}} = \num{34+0,6\%} $ и подставив ранее вычисленные значения в формулу~(\ref{eq:econ:calc_price_total:free_money:free}) получаем:
\begin{align}
  \label{eq:econ:calc_price_total:free_money:free_calc}
  \text{Э}_{\text{з}} &=
          \num{1,5} \cdot
          \text{(}
          \num{170} \cdot
          \num{12} -
          \num{24} \cdot
          \num{12}
          \text{)} \cdot
          \num{11} \cdot
          \text{(}\num{1} + \frac {\num{20\%}} {\num{100\%}}\text{)} \cdot
          \text{(}\num{1} + \frac {\num{34,6\%}} {\num{100\%}}\text{)} =\notag\\
          &= \SI{42692,201}{\text{\byr}}
          \text{\,.}
\end{align}

Прирост чистой прибыли рассчитывается по формуле~(\ref{eq:econ:calc_price_total:free_money:free_grouw}):
\begin{equation}
  \label{eq:econ:calc_price_total:free_money:free_grouw}
    \text{П}_{\text{ч}} =
    \sum_{i = 1}^{n}
    \text{Э}_\text{i} \cdot
    \text{(}
    \num{1} -
    \frac {\text{H}_{\text{n}}}
    {\num{100\%}}
    \text{)}
    \text{\,,}
\end{equation}
\begin{explanation}
  где & $ \text{n} $ & виды затрат, по которым получена экономия; \\
      & $ \text{Э} $ & сумма экономии, полученная за счет снижения \text{i}-ых затрат,~\byr; \\
      & $ \text{Н}_{\text{n}} $ & ставка налога на прибыль,~$ \% $.
\end{explanation}

Приняв ставку налога на прибыль $ \text{Н}_{\text{n}} = \num{18\%} $ и подставив ранее вычисленные значения в формулу~(\ref{eq:econ:calc_price_total:free_money:free_grouw}) получаем:
\begin{equation}
  \label{eq:econ:calc_price_total:free_money:free_grouw_calc}
  \text{П}_{\text{ч}} =
    \num{42692,201} \cdot
    \text{(}
    \num{1} -
    \frac {\num{18\%}}
    {\num{100\%}}
    \text{)}
    = \SI{34007,605}{\text{\byr}}
    \text{\,.}
\end{equation}

\subsubsection{Расчет прироста амортизационных отчислений}
\label{sub:econ:calc_price_total:amort}
, которые являются источником погашения инвестиций в приобретение программного продукта. Расчет амортизационных отчислений осуществляется по формуле~(\ref{eq:econ:calc_price_total:amort:amort_mf}):
\begin{equation}
  \label{eq:econ:calc_price_total:amort:amort_mf}
    \text{А} =
    \frac {\text{H}_{\text{а}} \cdot \text{И}_{\text{об}}}
    {\num{100\%}}
    \text{\,,}
\end{equation}
\begin{explanation}
  где & $ \text{Н}_{\text{а}} $ & норма амортизации программного продукта,~$ \% $; \\
      & $ \text{И}_{\text{об}} $ & стоимость программного продукта,~\byr.
\end{explanation}

Приняв норму амортизации программного продукта $ \text{Н}_{\text{а}} = \num{20\%} $ и подставив ранее вычисленные значения в формулу~(\ref{eq:econ:calc_price_total:amort:amort_mf}) получаем:
\begin{equation}
  \label{eq:econ:calc_price_total:amort:amort_mf_calc}
  \text{А} =
    \frac {\num{20\%} \cdot \num{122839,026}}
    {\num{100\%}}
    = \SI{24567,805}{\text{\byr}}
    \text{\,.}
\end{equation}

\subsection{Расчет показателей экономической эффективности проекта}
\label{sub:econ:efective}

При оценке эффективности инвестиционных проектов необходимо осуществить приведение затрат и результатов, полученных в разные периоды времени, к  расчетному году,  путем умножения затрат и результатов на коэффициент дисконтирования , который определяется по формуле~(\ref{eq:econ:efective}):
\begin{equation}
  \label{eq:econ:efective:efectiveamort_mf}
    \text{a}_{\text{t}} =
    \frac {\num{1}}
    {\text{(}\num{1} + \text{E}_{\text{н}}\text{)}^{\text{t}-\text{t}_{\text{p}}}}
    \text{\,,}
\end{equation}
\begin{explanation}
  где & $ \text{Е}_{\text{н}} $ & требуемая норма дисконта,~$ \% $; \\
      & $ \text{И}_{\text{об}} $ & порядковый номер года, затраты и результаты которого приводятся к расчетному году; \\
      & $ \text{t}_{\text{p}} $ & расчетный год, в качестве расчетного года принимается год вложения инвестиций, равный \num{1}.
\end{explanation}

Приняв норму дисконта $ \text{Е}_{\text{н}} = \num{25\%} $ и подставив ранее вычисленные значения в формулу~(\ref{eq:econ:efective:efectiveamort_mf}) получаем нормы дисонтирования за четыре года:
\begin{equation}
  \label{eq:econ:efective:efectiveamort_mf_calc}
    \text{a}_{\text{t\num{1}}} =
    \frac {\num{1}}
    {\text{(}\num{1} + \num{0,25}\text{)}^{\num{1}-\num{1}}} = \SI{1}{\text{}}
    \text{\,.}
\end{equation}

\begin{equation}
  \label{eq:econ:efective:efectiveamort_mf_calc}
    \text{a}_{\text{t\num{2}}} =
    \frac {\num{1}}
    {\text{(}\num{1} + \num{0,25}\text{)}^{\num{2}-\num{1}}} = \SI{0,80}{\text{}}
    \text{\,.}
\end{equation}

\begin{equation}
  \label{eq:econ:efective:efectiveamort_mf_calc}
    \text{a}_{\text{t\num{3}}} =
    \frac {\num{1}}
    {\text{(}\num{1} + \num{0,25}\text{)}^{\num{3}-\num{1}}}
    = \SI{0,64}{\text{}}
    \text{\,.}
\end{equation}

\begin{equation}
  \label{eq:econ:efective:efectiveamort_mf_calc}
    \text{a}_{\text{t\num{4}}} =
    \frac {\num{1}}
    {\text{(}\num{1} + \num{0,25}\text{)}^{\num{4}}}
    = \SI{0,51}{\text{}}
    \text{\,.}
\end{equation}

Расчет чистого дисконтированного дохода за четыре года реализации проекта и срока окупаемости инвестиций представлены в таблице~\ref{table:econ:efective:summary}:
\begin{table}[h!]
\caption{Экономические результаты работы предприятия}
\label{table:econ:efective:summary}
\centering
  \begin{tabular}{| >{\raggedright}m{0.205\textwidth}
                  | >{\centering}m{0.09\textwidth}
                  | >{\centering}m{0.06\textwidth}
                  | >{\centering}m{0.115\textwidth}
                  | >{\centering}m{0.115\textwidth}
                  | >{\centering}m{0.115\textwidth}
                  | >{\centering\arraybackslash}m{0.115\textwidth}|}
    \hline
      \multirow{3}{0.20\textwidth}{\centering Показатели} &
      \multirow{3}{0.09\textwidth}{\centering Ед. измер.} &
      \multirow{3}{0.06\textwidth}{\centering Усл. обоз.} &
      \multicolumn{4}{c|}{\centering Значения показателей по шагам} \\

    \cline{4-7}
    & & & & & & \\
    & & & $ t_0 $ & $ t_1 $ & $ t_2 $ & $ t_3 $ \\

    \hline
    Прирост чистой прибыли & \byr & $\Delta\text{П}_{\text{ч}}$ & \num{17503,8} & \num{35007,6} & \num{35007,6} & \num{35007,6} \\

    \hline
    Прирост амортизационных отчислений & \byr & $ \Delta\text{A} $ & \num{24567,8} & \num{24567,8} & \num{24567,8} & \num{24567,8} \\

    \hline
    Прирост результата & \byr & $\Delta\text{Р}_{\text{t}}$ & \num{42071,6} & \num{59575,4} & \num{59575,4} & \num{59575,4} \\

    \hline
    Коэффициент дисконтирования & & $\text{a}_{\text{t}}$ & \num{1} & \num{0,8} & \num{0,64} & \num{0,51} \\

    \hline
    Результат с учетом фактора времени & \byr & $\text{P}_{\text{t}}\text{a}_{\text{t}}$ & \num{42071,6} & \num{47660,3} & \num{38128,3} & \num{33183,5} \\

    \hline
    Инвестиции & \byr & $ \text{И}_{\text{об}} $ & \num{122839,1} & & & \\

    \hline
    Инвестиции с учетом фактора времени & \byr & $\text{И}_{\text{t}}\text{a}_{\text{t}}$ & \num{122839,1} & & & \\

    \hline
    Чистый дисконтированный доход по годам & \byr & $\text{ЧДД}_{\text{t}}$  & \num{-80767,4} & \num{47660,3} & \num{38128,3} & \num{30383,5} \\

    \hline
    ЧДД нарастающим итогом & \byr & $\text{ЧДД} $ & \num{-80767,4} & \num{-33107,1} & \num{5021,2} & \num{35404,6} \\

    \hline

  \end{tabular}
\end{table}

Рассчитаем рентабельность инвестиций $\text{Р}_{\text{и}}$ по формуле~(\ref{eq:econ:efective:renta}):
\begin{equation}
  \label{eq:econ:efective:renta}
    \text{Р}_{\text{и}} =
    \frac {\text{П}_{\text{чср}}}
    {\text{З}} \cdot
    \num{100\%}
    \text{\,,}
\end{equation}
\begin{explanation}
  где & $ \text{З} $ & затраты на приобретения нашего программного продукта,~\byr; \\
      & $ \text{П}_{\text{чср}} $ & среднегодовая величина чистой прибыли за расчетный период, \byr, которая определяется по формуле~(\ref{eq:econ:efective:sumpays}).
\end{explanation}

\begin{equation}
  \label{eq:econ:efective:sumpays}
    \text{П}_{\text{чср}} =
    \frac {\sum_{i = 1}^{n} \cdot \text{П}_\text{чt}}
    {\text{n}}
    \text{\,,}
\end{equation}
\begin{explanation}
  где & $ \text{П}_{\text{чt}} $ & чистая прибыль, полученная в году $\text{t}$,~\byr.
\end{explanation}
\newpage
Подставив данные в формулу~(\ref{eq:econ:efective:sumpays}) получаем среднегодовая величина чистой прибыли:
\begin{align}
  \label{eq:econ:efective:sumpays_calc}
  \text{Ц}_{\text{п}} &=
  \frac {\num{17503,802} + \num{35007,605} + \num{35007,605} + \num{35007,605}}
  {\num{4}} =\notag\\
  &= \SI{30631,654}{\text{\byr}} \text{\,.}
\end{align}

Подставим полученные данные в формулу~(\ref{eq:econ:efective:renta}):
\begin{equation}
  \label{eq:econ:efective:renta_calc}
    \text{Р}_{\text{и}} =
    \frac {\num{30631,654}}
    {\num{122839,026}} \cdot
    \num{100\%}
  = \SI{25}{\text{\%}} \text{\,.}
\end{equation}

В результате технико-экономического обоснования инвестиций по производству нового изделия были получены следующие значения показателей их эффективности:
\begin{itemize}
  \item чистый дисконтированный доход за четыре года производства продукции составит \num{35404,631}~\byr;
  \item треть инвестиций окупаются на четвертый год;
  \item рентабельность инвестиций составляет \num{25\%}.
\end{itemize}

Таким образом, внедрение программного продукта «Сетевой менеджер паролей с шифрованием/дешифрованием данных» для управления и хранения пользовательских данных, является эффективным и инвестиции в его разработку целесообразны с позиции прибыли.