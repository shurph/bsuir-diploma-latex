\section{Моделирование предметной области}
\label{sec:model}

\subsection{Модель социальных сил в моделировании пешеходных потоков}
\label{sec:model:sf}

В данном разделе будет более подробно описана оригинальная модель социальных сил.
Также в данном разделе предложены модификации модели социальных и рассмотрены дополнительные социальные силы, используемые в разрабатываемом ПС.

\subsubsection{Общее описание модели социальных сил}
\label{sec:model:sf:description}

Основной концепцией в модели социальных сил является абстрактное понятие социальной силы.
Под социальной силой понимается мера мотивации пешехода двигаться в определенном направлении.
Таким образом, социальная сила представляет собой направленный вектор.
Итоговое направление и некоторая мера скорости движения определяется как векторная сумма всех социальных сил, воздействующих на человека.

В модели социальных сил рассматривается две социальных силы, без которых модель не была бы корректной: движущая сила и сила отталкивания.
Движущая сила более подробно рассмотрена в разделе~\ref{sec:model:sf:moving_force}, а отталкивающая сила "--- в разделе~\ref{sec:model:sf:repulsion_force}.

В то время как направление движения однозначно определяется векторной суммой всех социальных, было бы не правильно однозначно определять скорость по ее модулю.
Вместо этого, в модель введено понятие желаемой скорости. Более подробно данная концепция описана в разделе~\ref{sec:model:sf:desired_speed}.

Важным моментом является учет поля зрения пешехода. В разделе~\ref{sec:model:sf:fov} описано как оригинальное решение авторов модели, так и решение, используемое в разрабатываемом ПС.

Также стоит упомянуть о концепции флуктуаций, которая описана в разделе~\ref{sec:model:sf:fluctuation}

Последним вопросом, рассматриваемым в разделе~\ref{sec:model:sf:panic}, является моделирование паники с помощью модели социальных сил.

\subsubsection{Движущая сила в модели социальных сил}
\label{sec:model:sf:moving_force}

Движущая сила представляет побуждение пешехода достичь своей цели.
Она всегда направлена к цели, а ее модуль зависит от желаемой скорости передвижения.
В модели сделано предположение, что желаемая скорость пешеходов распределена нормально со средним значением в $1.34$ м/с и среднеквадратичным отклонением в $0.26$ м/с.

\subsubsection{Отталкивающая сила в модели социальных сил}
\label{sec:model:sf:repulsion_force}

Сила отталкивания представляет побуждение пешехода сохранять некоторую дистанцию до других пешеходов и препятствий (стен).
Она направлена в противоположную от ближайшей точки препятствия сторону, а ее модуль в общем случае обратно зависит от расстояния до препятствия.
Авторы оригинальной модели социальных сил предлагают ввести поправочный коэффициент для учета анизотропности данной силы: пешеход держит большую дистанцию спереди и сзади, и меньшую – с боков.
Таким образом, модуль данной силы зависит не только от расстояния, но еще и от направления.

\subsubsection{Поддержание желаемой скорости в модели социальных сил}
\label{sec:model:sf:desired_speed}

\subsubsection{Учет поля зрения пешехода в модели социальных сил}
\label{sec:model:sf:fov}

Для большинства социальных сил имеет значение, мог ли данный пешеход видеть источник данной силы.
Для решения данной проблемы авторы оригинальной модели ввели еще один коэффициент, зависящий от угла между направлением движения и источником силы.
Данный коэффициент равен 1, если угол между направлением движения и источником силы по модулю меньше некоторого заданного значения (в работе использовалось значение 100 градусов), или 0.5 в обратном случае.
Очевидным улучшением предложенной модели будет использование не дискретного порога, а некоторой непрерывной функции.
Для наглядности эту функцию можно представлять в виде некоторой формы на двумерной плоскости.
Интуитивно понятно, что данная форма представляет собой форму поля зрения пешехода.
Значение весовой функции определяется как длина вектора от центра координат к точке пересечения луча, проведенного из начала координат под соответствующим углом.

Рисунок

\subsubsection{Флуктуации в модели социальных сил}
\label{sec:model:sf:fluctuation}

Для более реалистичного поведения к итоговой сумме всех сил добавляются некоторые случайные флуктуации.
Основой флуктуаций с социальной точки зрения являются неучтенные мотивации или случайные импульсные решения пешехода.
Также добавление флуктуаций позволяет выходить из «тупиковых» ситуаций, когда сумма всех сил близка по модулю к нулю.

\subsubsection{Моделирование паники с использованием модели социальных сил}
\label{sec:model:sf:panic}

\subsection{Функциональная спецификация разрабатываемого ПС}
\label{sec:model:func_spec}
