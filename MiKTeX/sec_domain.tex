\section{Разработка общей структуры микро-ЭВМ}
\label{sec:domain}

В данном разделе приводится функциональный состав разрабатываемой микро-ЭВМ, разработка и описание системы команд, а также описание взаимодействия всех блоков при выполнении команд программы.

\subsection{Функциональный состав микро-ЭВМ}
Для правильной работы и выполнения операций микро-ЭВМ должна обладать определенным набором функциональных блоков. Этот набор определяется архитектурой разрабатываемой микро-ЭВМ и требованиями, которые предъявляются при проектировании. Опишем набор функциональных блоков микро-ЭВМ для нашего варианта.

Наша микро-ЭВМ состоит из семи основных блоков: устройство управления, блок выборки инструкций, блок выборки операндов, блок выполнения, блок регистров, блок записи результата, память.

Устройство управления выставляет сигналы, необходимые для начала работы других блоков, устанавливает максимальное количество тактов работы для каждого устройства.

Блок выборки инструкций осуществляет выборку и декодирование команд. Этот блок включает в себя блок выборки, блок декодирования. Блок выборки в свою очередь содержит программный счетчик, который указывает на текущую команду в памяти команд. Блок декодирования принимает из памяти команд очередную команду и осуществляет разделение команды на отдельные составляющие (код операции, адреса регистров, памяти и т.д.).

Блок выборки операндов осуществляет загрузку операндов из регистров или RAM.

Блок исполнения предназначен для исполнения команд, поступающих из блока выборки операндов.
Он содержит блоки: стек, арифметико-логическое устройство (АЛУ). АЛУ хранит регистр флагов, который используется при переходе JS.

Блок записи результата осуществляет запись данных в РОН или ОЗУ.

Память в нашей архитектуре разделяется на два модуля.
Первый модуль является памятью команд, который представляет собой синхронную ROM,
тактируемую процессором.
Второй предназначен для хранения данных, над которыми производятся операции.
Память данных является синхронной RAM, и её тактирование также осуществляется процессором.
Оба модуля имеют размер в 1024 слов по два байта, так как размер шины адреса по условию равен 11 бит, а старший бит определяет какому модулю принадлежит адрес. Это позволяет иметь общее адресное пространство для RAM и ROM, так как микро-ЭВМ построена на Принстонской архитектуре. Если старший бит -- 1, то младшие 10 бит -- адрес RAM, в противном случае -- это адрес ROM.

Блок РОН содержит 10 шестнадцатиразрядных регистров, доступ к которым осуществляется за один такт процессорного времени, что позволяет значительно ускорить время обработки данных. Стек содержит 11 шестнадцатиразрядных регистров, объединенных в структуру доступа LIFO (last in first out), что может пригодиться в некоторых операциях обработки данных. Шестнадцатиразрядное АЛУ предназначено для выполнения четырех различных операций, что является главной задачей процессора.

\subsection{Разработка системы команд}
Для того, чтобы процессор мог выполнять определенные действия, необходимые для правильной обработки данных, должен быть разработан четкий набор инструкций, с помощью которого можно записать алгоритм на понятном процессору языке. В таблице 1.1 приведены все команды и способы их кодирования, разработанные в соответствии с заданным вариантом.

Как видно из таблицы 1.1  разрабатываемая микро-ЭВМ «понимает» 16 команд, с помощью которых можно построить различные алгоритмы обработки данных.

Процессор спроектированной системы поддерживает следующие типы адресации в командах АЛУ:
\begin{itemize}
    \item Прямая – адресация, при которой адрес в памяти указывается непосредственно в команде.
    \item Косвенная регистровая – номер регистра, в котором хранится адрес ячейки памяти, указывается непосредственно в команде
\end{itemize}
Также присутствуют команды MOV, позволяющие пересылать данные между регистрами, регистром и памятью, и загружать данные в регистр непосредственно из операнда команды.

Операции PUSH, POP предназначены для работы со стеком и работают с регистрами.

Операции перехода позволяют совершать безусловные переходы и переходы по условию. Операция JS совершает переход в случае, если установлен флаг S (флаг знака, означает, что старший бит результата АЛУ равен 1).
Операция JMP осуществляет безусловный переход.

И, наконец, операция HLT завершает работу процессора, сбрасывает регистры КЭШ и останавливает тактирование.

\begin{table}[ht]
\caption{Список команд микро-ЭВМ}
\label{table:domain:learning:number_of_models}
\centering
  \begin{tabular}{| >{\centering}m{0.1\textwidth}
                  | >{\raggedright}m{0.22\textwidth}
                  | >{\centering}m{0.14\textwidth}
                  | >{\centering}m{0.14\textwidth}
                  | >{\centering}m{0.13\textwidth}
                  | >{\centering\arraybackslash}m{0.11\textwidth}|}
      \hline КОП & Мнемоническая запись команды & Бит[26-16] & Бит[15-12] & Бит[11-8] & Бит[7-4] \\
      \hline 0000 & DEC [reg] & -- & КОП & -- & [reg] \\
      \hline 0001 & AND reg, [reg] & -- & КОП & reg & [reg] \\
      \hline 0010 & NAND reg, [reg] & -- & КОП & reg & [reg] \\
      \hline 0011 & ROR reg, [reg] & -- & КОП & reg & [reg] \\
      \hline 0100 & MOV reg, \$mem & \$mem & КОП & reg & -- \\
      \hline 0101 & PUSH reg & -- & КОП & reg & --  \\
      \hline 0110 & JMP \$mem & \$mem & КОП & -- & -- \\
      \hline 0111 & MOV reg, [reg] & -- & КОП & reg & [reg] \\
      \hline 1000 & DEC reg & -- & КОП & reg & -- \\
      \hline 1001 & AND reg, reg & -- & КОП & reg & reg \\
      \hline 1010 & NAND reg, reg & -- & КОП & reg & reg \\
      \hline 1011 & ROR reg, reg & -- & КОП & reg & reg  \\
      \hline 1100 & MOV \$mem, reg & -- & КОП & reg & -- \\
      \hline 1101 & POP reg & -- & КОП & reg & -- \\
      \hline 1110 & JS \$mem & \$mem & КОП & -- & -- \\
      \hline 1111 & HLT & - & КОП & -- & -- \\
      \hline
  \end{tabular}
\end{table}

В таблице \$mem означает, адрес ячейки памяти. Первый операнд (биты[11-8]) всегда адрес регистра, второй -- адрес ячейки памяти, регистр, или регистр, служащий для косвенной адресвации.
Если второй операнд -- адрес ячейки памяти, то он хранится в младших битах второго слова.
Если он адрес регистра -- хранится в битах[7-4].

Все команды имеют длину в два слова, операнды могут отсутствовать.
Первое и второе слово дополняются нулями.

\subsection{Описание взаимодействия всех блоков микро-ЭВМ при выполнении команд программы}
Исполнение команды начинается с подачи тактового импульса на устройство управления. Устройство управления предоставляет фиксированное количество тактов для работы каждому устройству.

Затем вступает в работу блок декодирования команд. Он производит выборку команд, и их декодирование. Также он содержит счетчик инструкций, значение которого может быть изменено либо командой условного перехода, либо автоматически. Для выполения перехода блок включается на стадии выполнения команды.

Декодированная команда затем поступает на блок выборки операндов. Он производит чтение операндов из регистров и памяти, декодированные операнды поступают на блок исполнения.

Блок исполнения представляет собой АЛУ. Он также содержит стек. Постувшие операнды поступают либо на один из блоков АЛУ, либо в стек. Если над операндами не нужно выполнять действия -- по умолчанию пропускается на выход второй операнд. АЛУ также содержит регистр флагов, который хранит знак результата. Если провшлая команда установила этот флаг в 1 и пришла команда js -- выполняется переход.

Блок записи результатов записывает операнд либо в память, либо в регистр в зависимости от метода адресации операднов в команде.

Блок РОН участвует в операциях выборки операндов, записи результата.

Память содержит в едином адресном пространстве как RAM, так и ROM. На этапе декодирования команды используется ROM. При выборке операндов и записи результа -- RAM.

Стек доступен через команды push и pop. В случае ошибки при работе со стеком, стек выставляет сигналы empty и full.
