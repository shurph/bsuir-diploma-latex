\section{Функциональное проектирование} % (fold)
\label{sec:arch_and_mod}

\subsection{Описание конфигурации сборки проекта}
\label{sub:arch_and_mod:graphlib}

В отличие от большинства сборщиков проектов, приложение \npm{}\footnote{\url{https://www.npmjs.com/}} использует декларативный способ сборки проектов вместо императивного. Это значит, что \json{} файл содержит описание проекта, согласно которому данный проект собирается в исполняемый модуль платформы \nodejs{}.

В процессе построения проекта происходят следующие этапы:
\begin{itemize}
  \item синтаксический анализ – проверка на содержание файлом всех необходимых данных и на синтаксическую правильность. В данном пункте проверяется корректность декларативного описанию путем проверки с использованием \npmrc{};
  \item обработка ресурсов и создание основы приложения, загрузка из сети недостающих элементов, описанных в зависимостях;
  \item компиляция исходных файлов проекта, посредством \tsc{} и \typings{}\footnote{\url{https://github.com/typings/typings/}};
  \item обработка и компиляция тестов;
  \item тестирование – запуск тестов, только при успешно пройденных тестах выполняется следующий этап;
  \item формирование исполнительного модуля \nodejs{};
\end{itemize}

Тестовая среда, создаваемая после сборки проекта в исполняемый модуль, имеет все описанные в \json{} файле настройки. Файл package.json располагается в корневой директории проекта и является основным источником информации при запуске команд продукта \npm{}.

Общий вид файла, описывающий целый проект, содержит всю необходимую универсальному декларативному сборщику проектов \npm{} для формирования исполняемого модуля NodeJS и настройки тестовой среды для модуля расширения согласно описанных в файле настроек. Конфигурация исполнительного модуля приведена в листинге~\ref{lst:arch_and_mod:graph_definition}:
\begin{lstlisting}[style=json,caption={Конфигурация исполнительного модуля}, label=lst:arch_and_mod:graph_definition]
{
  "name": "password-manager",
  "description": "Password-manager of Artem Derevnjuk",
  "version": "0.1.2-rc.3",
  "main": "dist/passman.js",
  "author": "Artem Derevnjuk",
  "repository": {
    "type": "git",
    "url": "https://github.com/derevnjuk/passman.git"
  },
  "bugs": {
    "url": "https://github.com/cderevnjuk/passman /issues"
  },
  "keywords": [
    "passman",
    "password",
    "manager",
    "utility"
  ],
  "dependencies": {
    "body-parser": "^1.15.0",
    "express": "^4.13.4",
    "jsonwebtoken": "^5.7.0",
    "mongoose": "^4.4.8",
    "nconf": "^0.8.4",
    "promise": "^7.1.1",
    "node-uuid": "^1.4.2",
    "clone": "^0.1.19",
    "crypto-js": "^3.1.2",
  },
  "devDependencies": {
    "chai": "^3.1.0",
    "coveralls": "^2.11.2",
    "es6-promise": "^2.3.0",
    "fs-extra": "^0.26.7",
    "jscs": "^1.13.1",
    "jscs-jsdoc": "^1.3.2",
    "jshint": "~2.8.0",
    "karma": "^0.13.2",
    "karma-browserify": "^4.2.1",
    "karma-firefox-launcher": "^0.1.6",
    "karma-mocha": "^0.2.0",
    "karma-mocha-reporter": "^1.0.2",
    "mocha": "^2.2.5",
    "native-promise-only": "^0.8.0-a",
    "nodeunit": ">0.0.0",
    "nyc": "^2.1.0",
    "recursive-readdir": "^1.3.0",
    "rimraf": "^2.5.0",
    "rollup": "^0.25.0",
    "rollup-plugin-node-resolve": "^1.5.0",
    "rollup-plugin-npm": "~1.3.0",
    "rsvp": "^3.0.18",
    "semver": "^4.3.6"
  },
  "scripts": {
    "coverage": "nyc npm test && nyc report",
    "coveralls": "nyc npm test && nyc report --reporter=text-lcov | coveralls",
    "lint": "jshint lib/ test/ mocha_test/ perf/memory.js perf/suites.js perf/benchmark.js support/ karma.conf.js && jscs lib/ test/ mocha_test/ perf/memory.js perf/suites.js perf/benchmark.js support/ karma.conf.js",
    "mocha-browser-test": "karma start",
    "mocha-node-test": "mocha mocha_test/ --compilers js:babel-core/register",
    "mocha-test": "npm run mocha-node-test && npm run mocha-browser-test",
    "nodeunit-test": "nodeunit test/test-async.js",
    "test": "npm run-script lint && npm run nodeunit-test && npm run mocha-node-test"
  },
  "license": "MIT",
  "jam": {
    "main": "dist/passman.js",
    "include": [
      "dist/passman.js",
      "README.md",
      "LICENSE"
    ],
    "categories": [
      "Utilities"
    ]
  },
  "spm": {
    "main": "dist/passman.js"
  },
  "volo": {
    "main": "dist/passman.js",
    "ignore": [
      "**/.*",
      "node_modules",
      "bower_components",
      "test",
      "tests"
    ]
  }
}
\end{lstlisting}

Данный файл для сборщика \npm{} преобразуется неявно в совокупность свойств. Описание основных свойств файла программной модели описано ниже:
\begin{itemize}
  \item name, значение определяет уникальное имя пакета, пространство имен, общее для всех единиц исполняемого модуля;
  \item description, описание пакета, предоставляет исчерпывающую информацию по исполняемому модулю при его публикации;
  \item version, версия исполняемого модуля;
  \item main, точка входа в приложение, в тоже время идентификационное имя исполняемого модуля;
  \item author, информация об авторе и поставщике исполняемого модуля для платформы. Содержит такую информацию, как имя автора и ссылка на более подробную информацию о разработчике модуля;
  \item repository, публичный адрес на репозиторий в системе контроля версий,
  описывающий порядок доступа к проекту приложения;
  \item bugs, связывает bugs-tracking с системами автоматизированного обслуживания и отчетности;
  \item keywords, список ключевых слов для поиска пакета в менеджере пакетов;
  \item dependencies, список зависимостей, необходимых для корректного запуска приложения, набор описаний зависимых компонент модуля, описываются загружаемые в момент сборки компоненты, обеспечивающих основу модуля при сборке.
  \item devDependencies, список параметров пакетов рабочего окружения, который определяет набор исполняемых пакетов сборщик NPM, GULP и SystemJS, необходимых для успешной сборки исполняемого модуля;
  \item scripts, словарь, содержащий скриптовые команды, которые запускаются во время жизни приложения по установленному условию;
  \item jam, список клиентских зависимостей, динамически внедряемых пакетов, необходимых для успешного запуска клиентского приложения.
  \item spm, статический идентификатор в системе пакетов SPM.
  \item volo, конфигурация конструктора модулей  AMD, CommonJS, Node.
\end{itemize}

В данном исполняемом модуле используются следующие компоненты и зависимости, описанные в dependencies:
\begin{itemize}
  \item body-parser – межпрограммный модуль обработки тела запроса, ответственный за разбор POST запросов от клиента;
  \item express – реактивный северный-фраймворк для \nodejs{}, предоставляющий основные элементы взаимодействия с функциональностью на стороне клиента посредством протокола SOAP. Содержит основной набор классов, необходимый для реализации REST-сервиса;
  \item jsonwebtoken – компонент, обеспечивающий аутентификацию по методу JSON Web Token\footnote{\url{https://jwt.io/}} со средствами хеширования JWT маркера;
  \item mongoose – система, которая предоставляет возможность объектно-реляционного отображения объектной модели приложения, в соответствии с интерфейсом NodeJS.
  \item nconf – основной компонент динамической конфигурации сервера на базе объектной модели. Является наиболее часто используемым компонентов взаимодействия сервера с нижестоящими слоями;
  \item promise – компонент спецификации ES2016\footnote{\url{https://github.com/tc39/ecma262}}, предоставляющий одни из рекомендуемых способов организации асинхронного кода
  \item node-uuid — модуль авторизации, поддерживающий JWT маркеры.
  \item clone — стандартный компонент многопроцессорного взаимодействия, производящий обмен данных между объектной моделью приложения и базой данных
  \item crypto-js — библиотека содержит реализацию функций шифрования и хеширования на северной и клиентской стороне.
\end{itemize}

В файл программной модели модуля могут также быть добавлены дополнительные свойства. При выполнении консольной команды npm run происходит полный цикл построение программного модуля, запуск тестов и запуск сервера \nodejs{} для работы тестовой среды приложения.
На программный модуль при использовании сборщика NPM накладывается ряд ограничений. Запуск команды npm run приводит к созданию в корневом каталоге папок node\_modules и typings, содержимое которых заполняется по мере выполнения команды. После успешного выполнения в папке node\_modules можно найти зависимости приложения и модуля тестирования данного программного средства, а также файлы тестовой среды для исполнительного модуля. Которые представляют собой одиночную версию разрабатываемого продукта в режиме быстрой разработки. Данный режим сканирует любые изменения исходного кода модуля и изменяет по мере выполнения функциональность определенных в модуле элементов. Эта возможность позволяет не перезапускать команду лишний раз при наличии изменений.

\subsection{Классы Business Layer}
\label{sub:arch_and_mod:probab_net}

К классам Business Layer относятся классы, наследующие от класса EventEmittor, который находится в пространстве имен NodeJS.events.EventEmit"=tor. Основной ролью данных классов является формирования логики обработки всех декларированных объектов объектной модели во время жизненного цикла приложения в рамках асинхронной модели ввода\/вывода.

\subsubsection{Класс ObserverServer }
\label{sub:arch_and_mod:probab_net:observer}

является базовым абстрактным классом для классов бизнес логики и наследует свойства и методы абстрактного класса EventEmittor. Класс не содержит дополнительных атрибутов и модификаторов доступа.

Поля:
\begin{itemize}
  \item \_logger – поле типа Logger, используется для логирования исключительных ситуаций, сбоев и отсутствие разрешений;
  \item applicationProperties – поле типа ApplicationProperties, позволяющее получить свойства текущей конфигурации ObserverServer;
  \item \_dataResultFactory – поле типа IDataResult, возвращающее выборку фабрики результатов, формирует модель результат валидации для дальнейшей передачи на слой выше.
\end{itemize}

Методы:
\begin{itemize}
  \item IObserverServer ObserverServer(ApplicationProperties applicationProper"=ties) – конструктор класса, принимающий в качестве параметров объект
  \item конфигурации;
  \item boolean \_hasAdminPermission() – метод, определяющий наличие прав на администрирование пользователей и групп. Метод используется для выполнения запросов к базе данных, предотвращая несанкционированные попытки получить пользовательские данные посредством не авторизованного доступа.
\end{itemize}

\subsubsection{Класс Westley }
\label{sub:arch_and_mod:probab_net:westley}

представляет собой класс, реализующий функционал редактора хранилища и менеджер истории его изменений. Является реализацией интерфейса IWestley.

Поля:
\begin{itemize}
  \item \_dataset – поле типа IStorageData исполнения процедуры в базе данных. На вход принимает модель запроса, на выходе возвращает результат выполнения процедуры;
  \item \_history – поле типа Array<IStorageHistory> контейнер истории для текущего пользователя;
  \item \_cachedCommands – поле типа Map<Command>, содержащая результаты наиболее частых запросов к хранилищу.
\end{itemize}

Методы:
\begin{itemize}
  \item IWestley clear() – метод возвращающий пустой объект хранилища.
  \item IWestley execute(Command command) – метод, принимает команды типа Command и делегирует их выполнение объекту command"=Tools типа ICommandTools, сохраняя историю и изменяя \_dataset.
  \item Command \_getCommadForKey(string commandKey) – метод возвращает command типа Command по переданному ключу из свойства \_cached"=Commands.
  \item IWestley pad() – метод возвращающий хранилище, внедряя смещение в оригинальный прототип хранилища.
  \item IStorageData getDataset() – метод возвращает результат выполнения процедуры в базе данных.
  \item Array<IStorageHistory> getHistory() – метод возвращает контейнер истории запросов к хранилищу.
\end{itemize}

\subsubsection{Класс Archive }
\label{sub:arch_and_mod:probab_net:archive}

управляет временем существования компоновкой и обработкой объектов ManagedEntry и ManagedGroup. Данный класс может иметь только один экземпляр для конкретного пользователя.

Поля:
\begin{itemize}
  \item \_westley – поле типа IWestley, инкапсулирует редактор хранилища и менеджер истории запросов к нему;
  \item date – поле типа Date, содержит дату создания экземпляра класса Archive.
\end{itemize}

Методы:
\begin{itemize}
  \item Archive() – конструктор класса;
  \item Array<ManagedEntry> findEntriesByCheck(Achive achive, string check, string key, RegExp value) – статический метод, возвращающий массив всех записей экземпляра класса Archive на основании переданных мета-значений;
  \item boolean containsGroupWithTitle(string groupTitle) – метод сопоставления всех существующих групп переданному заголовку группы;
  \item IManagedGroup createGroup(string title) – метод принимает имя группы в качестве аргумента и возвращает ее имплементацию;
  \item Array<IManagedGroup> findEntriesByMeta(string metaName, RegExp value) – метод сбора релевантной выборки групп по переданным, в качестве параметров, мета-данным и их значению типа string, и RegExp соответственно;
  \item Array<IManagedGroup> findEntriesByProperty(string property, RegExp value) – метод сбора релевантной выборки групп по переданным, в качестве параметров, свойству и его значению типа string, и RegExp соответственно;
  \item Array<IManagedGroup> findGroupsByTitle(string title) – метод поиска всех групп архива, которые соответствуют преданному имени title;
  \item IManagedEntry getEntryByID(string entryID) – метод, осуществляющий поиск записи по ее уникальному идентификатору;
  \item IManagedGroup getGroupByID(string groupID) – метод, который производит рекурсивный поиск по идентификатору группы;
  \item Array<ManagedGroup> getGroups() – метод возвращает все группы, которые содержит экземпляр класса Archive;
  \item ManagedGroup getTrashGroup() – метод, позволяющий получить удавленные группы, подписанные на экземпляр класса Archive;
  \item IWestley \_getWestley() – метод возвращает базовый экземпляр типа IWestley;
  \item Achive createWithDefaults() – метод, возвращающий экземпляр класса Archive c параметрами по умолчанию.
\end{itemize}

\subsubsection{Класс InigoCommand }
\label{sub:arch_and_mod:probab_net:inigoCommand}

представляет собой класс для обработки и выполнения команд типа Command, наследующий от статического класса REPL.

Поля:
\begin{itemize}
  \item \_commandKey – поле типа string, которое содержит индекс выполняемой команды;
  \item \_commnadArgs – поле типа Array<strign arg> вмещает набор аргументов команды;
  \item commandArgument – статическое поле типа Map<ItemMetaRoot>, хранит маркеры и идентификаторы, присвоенные Command по умолчанию;
  \item commands – статическое поле типа Map<Command>, описывающее все подписанные команды в рамках текущей реализации CommandArguments.
\end{itemize}

Методы:
\begin{itemize}
  \item InigoCommand() – конструктор класса;
  \item InigoCommand addArgument(...args) – метод принимает неизвестное количество аргументов, привязывая их к текущей выполняемой команде, возвращает
  \item объект типа InigoCommand;
  \item Command generateCommand() – метод возвращающий реализацию команды на основании полей \_commandKey и \_commnadArgs;
  \item InigoCommand create(string cmd) – статический метод, возвращающий один и только один экземпляр класса InigoCommand для единственной команды по переданному индексу.
\end{itemize}

\subsubsection{Класс ManagedGroup }
\label{sub:arch_and_mod:probab_net:managedgroup}

представляет собой класс для управления группами записей пользователя, наследующий от стандартного класса Stream. Определяет порядок привязки групп к хранилищам пользователя и подписку на события глобальной группы. Является реализацией интерфейса IManagedGroup.

Поля:
\begin{itemize}
  \item \_archive – поле типа string, которое содержит индекс выполняемой команды;
  \item \_westley – поле типа Array<string> вмещает набор аргументов команды;
  \item \_remoteObject – поле типа IremoteObject, содержащее ссылку на удаленную группу подписанную на события экземпляр ManagedGroup.
\end{itemize}

Методы:
\begin{itemize}
  \item ManagedGroup() – конструктор класса;
  \item ManagedEntry createEntry(string title) – метод создания новой записи с заголовком title, возвращающий экземпляр класса ManagedEntry;
  \item ManagedGroup createGroup(string title) – метод создания дочерний группы с именем title, возвращающий экземпляр класса;
  \item void delete() – метод удаления группы типа ManagedGroup и очистки \_westley и \_removeObject;
  \item ManagedGroup deleteAttribute(string attr) – метод удаления атрибута по переданному имени attr, возвращает ссылку на экземпляр класса
  \item ManagedGroup;
  \item string getAttribute(string attributeName) – метод принимает имя атрибута и возвращает значение релевантного атрибута типа string;
  \item Array<ManagedEntry> getEntries() – метод, возвращающий массив записей группы типа ManagedEntry, привязанной к экземпляру класса;
  \item Array<ManagedGroup> getGroup() – метод, возвращающий массив всех групп типа ManagedGroup привязанных к глобально группе;
  \item string getID() – метод возвращает идентификатор глобальной группы;
  \item string getTitle() – метод возвращает имя глобальной группы;
  \item boolean isTrash() – метод, осуществляющий проверку на тип группы, возвращает значение boolean;
  \item ManagedGroup moveToGroup(ManagedGroup group) – метод перемещения подчиненной группы в группу group, переданную в качестве параметра, возвращает ссылку на экземпляр класса;
  \item ManagedGroup setTitle(string title) – метод, который принимая заголовок группы типа string, устанавливает ее в качестве имени глобальной группы;
  \item ManagedGroup setAttribute(string attributeName, string value) – метод принимает имя свойства и его значения, присваивая его глобальной группе;
  \item IArchive \_getArchive() – метод возвращает реализацию интерфейса IArchive – владельца глобальной группы;
  \item IRemoteObject \_getRemoteObject() – метод, который возвращает значение поля remoteObject текущего экземпляра класса;
  \item IWestley \_getWestley() – метод, который возвращает значение поля \_westley в контексте экземпляра класса;
  \item ManagedGroup createNew(IArchive archive, string parentID) – статический метод, позволяющий создать новый экземпляр класса ManagedGroup в хранилище archive с ведущей группой, обладающей parentID.
\end{itemize}

\subsubsection{Класс ManagedEntry }
\label{sub:arch_and_mod:probab_net:managedentry}

используется для инициализации записи пользователя. Наследует свойства и методы абстрактного стандартного класса BaseManagementEntry. Имеет шаблон \_remoteObject для представления и инициализации функциональности на стороне клиента.

Поля:
\begin{itemize}
  \item \_archive – поле типа string, которое содержит индекс выполняемой команды;
  \item \_westley – поле типа Array<string> вмещает набор аргументов команды;
  \item \_remoteObject – поле типа IRemoteObject, содержащее ссылку на скрытый объект-токен, содержащий хранимую информацию.
\end{itemize}

Методы:
\begin{itemize}
  \item ManagedEntry(Archive archive, IRemoteObject remoteObj) – конструктор класса, в качестве параметров принимает экземпляр archive и ссылку на удаленное представление remoteObj;
  \item void delete() – метод удаляет запись, в случае, если запись уже удалена, полностью очищает все ссылки на нее;
  \item ManagedEntry deleteAttribute(string attr) – метод, удаляющий параметр записи, переданный в качестве единственного аргумента attr. Возвращает ссылку на текущий экземпляр записи типа ManagedEntry;
  \item ManagedEntry deleteMeta(string property) – метод удаляет метаданные по переданному параметру property. Возвращает ссылку на экземпляр на котором был вызван;
  \item string getAttribute(string attr) – метод возвращает значение параметра записи по преданному имени атрибута;
  \item DisplayInfo getDisplayInfo() – метод возвращает экземпляр класса Dis"=playInfo, который представляя объект типа JSON;
  \item MangedGroup getGroup() – метод возвращает ссылку на группу типы ManagedGroup, содержащий экземпляр текущей записи;
  \item string getMeta(string property) – метод возвращает мета-данные по ключу property;
  \item string getID() – метод, возвращающий идентификатор записи в качестве строки;
  \item string getProperty(string property) – метод возвращает значение свойства, переданного в качестве первого аргумента;
  \item ManagedEntry moveToGroup(ManagedGroup group) – метод позволяет переместить запись в группу, ссылку на которую принимает в качестве первого параметра;
  \item ManagedEntry setAttribute(string name, string value) – метод присваивает атрибут с именем name и значением value записи, возвращая ссылку на нее;
  \item ManagedEntry setProperty(string name, string value) – метод присваивает свойство с именем name и значением value записи, возвращая ссылку на нее;
  \item IArchive \_getArchive() – метод возвращает реализацию интерфейса IArchive, содержащего все записи и группы;
  \item IRemoteObject \_getRemoteObject() – метод, который возвращает значение поля remoteObject текущего экземпляра класса;
  \item IWestley \_getWestley() – метод, который возвращает значение поля \_westley в контексте экземпляра класса;
  \item ManagedEntry createNew(IArchive archive, string parentID) – статический метод, позволяющий создать новый экземпляр класса ManagedGroup в хранилище archive с ведущей группой, обладающей parentID.
\end{itemize}

\subsubsection{Класс Credentials }
\label{sub:arch_and_mod:probab_net:credentials}

используется для формирования и проверки прав доступа к хранилищам. Наследует свойства и методы абстрактного стандартного класса BaseManagementEntry.

Поля:
\begin{itemize}
  \item \_signing – поле типа Array<string> вмещает набор аргументов команды;
  \item \_model – поле типа IRemoteObject, содержащее ссылку на скрытый объект-токен, содержащий хранимую информацию.
\end{itemize}

Методы:
\begin{itemize}
  \item Credentials(Model data) – конструктор класса, в качестве единственного параметра принимает экземпляр типа Model;
  \item Credentials setIdentity(string username, string password) – метод устанавливает username и password поля \_model текущего экземпляра. Возвращает экземпляр класса Credentials;
  \item Credentials setType (CredentialsType type) – метод присваивает полномочия типа type, переданного в качестве первого параметра, модели поля \_model;
  \item Promise<IOcane> convertToSecureContent(string password) – метод преобразующий учетные данные в шифрованную строку посредством токена password. Возвращает Promise типа IOcane;
  \item Promise<IOcane> createFromSecureContent(IOcane content, password) – статический метод, возвращающий экземпляр класса Credentials, содержащий дешифрованные данные content.
\end{itemize}

\subsubsection{Класс Certificate }
\label{sub:arch_and_mod:probab_net:certificate}

Поля:
\begin{itemize}
  \item \_GCM – поле типа Buffer, содержит дополнительную информацию, используемую в качестве дополнительно параметра при проверки подлинности подписи;
  \item \_generator – поле типа IdiffieHellman, вмещает ссылку на генератор  Диффи-Хеллмана в кодировке поля \_GCM;
  \item digest – поле типа Int32Array, словарь описывающий форму хеширования данных для экземпляров класса EncodeForfatter;
  \item \_level – поле типа number, отражает количество итераций сдвига случайных кривых;
  \item \_timestamp – поле типа Date, содержит расчетное значение времени необходимого для осуществления шифрования;
  \item \_crypto – поле типа ICrypto, ссылка на композицию функций шифрования и хеширования.
\end{itemize}

Методы:
\begin{itemize}
  \item Certificate() – конструктор класса;
  \item Buffer exportChallenge(ISuperToken spkac) – метод, принимает аргумент типа ISuperToken, который вмещает в себя открытый ключ и проверку типа Сhallenge. Возвращает объект типа Buffer, инкапсулирующий переданный ключ и тайну;
  \item boolean verifySpkac(Buffer spkac) – метод осуществляет проверку на соответствие тайны и публичного ключа;
  \item Buffer getAuthTag() – метод возвращает тег аутентификации типа Buffer, прошедшего проверку подлинности шифрования;
  \item Buffer computeSecret(ISuperToken spkac) – метод принимает аргумент типа ISuperToken, на основании которого производит вычисление секрета по протоколу Диффи-Хеллмана. Возвращает секрет типа Buffer;
  \item IDiffieHellman getGenerator() – метод возвращает \_generator типа  IDif"=fieHellman;
  \item Buffer getPrime() – метод возвращает большое простое число типа Buffer;
  \item Buffer getPrivateKey() – метод возвращает приватный ключ типа Buffer;
  \item Buffer getPublicKey() – метод возвращает публичный ключ типа Buffer;
  \item void mac(Uint8Array message, Int32Array hmac) – метод, осуществляющий проверку подлинности хеш-функции hmac посредством переданного ключа message типа UInt8Array;
  \item Int32Array createMAC(Buffer password) – метод, позволяющий создать хеш-функцию для переданного пароля password типа Buffer;
  \item Buffer keyBlock(Uint8Array password, Uint8Array salt, number iteration, number blockIndex) – метод вычисления блока ключа для переданной соли типа Uint8Array и пароля password типа Uint8Array. Возвращает blockIndex блок ключа на итерации iteration, номер которой передан третьим параметром;
  \item Buffer compress(string data) – метод осуществляет декрементальный разбор данных и их сжатие по словарю. Возвращает сжатые данные типа Buffer;
  \item string decompress(Buffer data) – метод производит восстановление сжатых данных по словарю с их последующей нормализацией и смещением преобразующей формы с относительным смещением в форму с абсолютным;
  \item string normalizTextValue(string data) – метод нормализует данные после восстановления. Возвращает строку типа string;
  \item string getUniqueID() – метод возвращает индикатор среды распределенных вычислений, определяя уникальный флаг хранилища пользователя;
  \item string hashText(string data) – метод, осуществляющий хеширование данных типа string, переданных ему в качестве перового параметра;
  \item Buffer decideImpl(number iteration, number  blockIndex) – метод нормализует секрет Диффи-Хеллмана. Возвращает секрет типа Buffer;
  \item Buffer encrypt(Buffer data, Buffer key) – метод производит шифрование сжатых данных data по ключу key;
  \item Buffer decrypt(Buffer ciphertext, Buffer pubkey) – метод производит дешифрование сжатых данных ciphertext по публичному ключу pubkey.
  \item number \_nbi() – метод возвращает число типа BigInteger.
\end{itemize}

\subsection{Классы Data Layer}
\label{sub:arch_and_mod:data_layer}

К классам Business Layer относятся классы, наследующие от класса EventEmittor, который находится в пространстве имен NodeJS.events.Event"=Emittor. Основной ролью данных классов является формирования логики обработки всех декларированных объектов объектной модели во время жизненного цикла приложения в рамках асинхронной модели ввода\/вывода.

\subsubsection{Класс AddGeneralDataAction }
\label{sub:arch_and_mod:data_layer:add_general_data_action}

представляет собой класс, реализующий добавление новых записей в хранилище. Является реализацией интерфейса IDataAction.

Поля:
\begin{itemize}
  \item \_applicationContext – поле контекста приложения, содержит в себе информацию о текущем пользователе, локализации;
  \item \_dataResultFactory – поле фабрики результатов, формирует модель результат валидации для дальнейшей передачи на клиент;
  \item \_dataSourceInvoker – поле исполнения процедуры в базе данных. На вход принимает модель запроса, на выходе возвращает результат выполнения процедуры;
  \item \_partDetailQuoteHelper – поле класса помощника, выполняющего проверку на разрешение редактирования записи, и контейнера констант;
  \item \_quoteWizardSettings – контейнер настроек для текущего пользователя.
\end{itemize}

Методы:
\begin{itemize}
  \item AddGeneralDataAction() – конструктор класса;
  \item Task<IDataResult<PartDetailFormModel, PartDetailFormModel>> Proces"=sAsync(PartDetailFormModel contextModel) – метод, выполняющий обработку данных и запись этих данных в таблицу. На вход поступает модель, которую необходимо обновить, на выходе получается результат добавления.
\end{itemize}

\subsubsection{Класс AddMiscItemDataAction }
\label{sub:arch_and_mod:data_layer:add_misc_item}

представляет собой класс, реализующий добавление новых записей в таблицу Miscellaneous хранилища. Является реализацией интерфейса IDataAction.

Поля:
\begin{itemize}
  \item \_applicationContext – поле контекста приложения, содержит в себе информацию о текущем пользователе, локализации;
  \item \_dataResultFactory – поле фабрики результатов, формирует модель результат валидации для дальнейшей передачи на клиент;
  \item \_dataSourceInvoker – поле исполнения процедуры в базе данных. На вход принимает модель запроса, на выходе возвращает результат выполнения процедуры;
  \item \_partDetailQuoteHelper – поле класса помощника, выполняющего проверку на разрешение редактирования записи, и контейнера констант;
  \item \_quoteWizardSettings – контейнер настроек для текущего пользователя;
  \item \_updatePartDetailFormDataHelper – помощник для работы с сервисом запросов базы данных.
\end{itemize}

Методы:
\begin{itemize}
  \item AddMiskItemDataAction() – конструктор класса;
  \item Task<IDataResult<PartDetailFormModel, PartDetailFormModel>> Proces"=sAsync(PartDetailFormModel contextModel) – метод, выполняющий обработку данных и запись этих данных в таблицу. На вход поступает модель, которую необходимо обновить, на выходе получается результат добавления.
\end{itemize}

\subsubsection{Класс AddPartComponent }
\label{sub:arch_and_mod:data_layer:add_part_component}

представляет собой класс, реализующий добавление новых записей типа KeyComponent. Является реализацией интерфейса IDataAction.

Поля:
\begin{itemize}
  \item \_applicationContext – поле контекста приложения, содержит в себе информацию о текущем пользователе, локализации;
  \item \_dataResultFactory – поле фабрики результатов, формирует модель результат валидации для дальнейшей передачи на клиент;
  \item \_partDetailQuoteHelper – поле класса помощника, выполняющего проверку на разрешение редактирования записи, и контейнера констант;
  \item \_dataSourceInvoker – поле исполнения процедуры в базе данных. На вход принимает модель запроса, на выходе возвращает результат выполнения процедуры;
  \item \_quoteWizardSettings – контейнер настроек для текущего пользователя;
  \item \_updatePartDetailFormDataHelper – помощник для работы с сервисом запросов базы данных.
\end{itemize}

Методы:
\begin{itemize}
  \item AddPartComponent() – конструктор класса;
  \item Task<IDataResult<PartDetailFormModel, PartDetailFormModel>> Proces"=sAsync(PartDetailFormModel contextModel) – метод, выполняющий обработку данных и запись этих данных в таблицу. На вход поступает модель, которую необходимо обновить, на выходе получается результат добавления.
\end{itemize}

\subsubsection{Класс AddPaymentsRoutingDataAction }
\label{sub:arch_and_mod:data_layer:add_payments_routing}

представляет собой класс, реализующий добавление новых записей типа PaymentsRouting. Является реализацией интерфейса IDataAction.

Поля:
\begin{itemize}
  \item \_applicationContext – поле контекста приложения, содержит в себе информацию о текущем пользователе, локализации;
  \item \_dataResultFactory – поле фабрики результатов, формирует модель результат валидации для дальнейшей передачи на клиент;
  \item \_dataSourceInvoker – поле исполнения процедуры в базе данных. На вход принимает модель запроса, на выходе возвращает результат выполнения процедуры;
  \item \_partDetailQuoteHelper – поле класса помощника, выполняющего проверку на разрешение редактирования записи, и контейнера констант;
  \item \_quoteWizardSettings – контейнер настроек для текущего пользователя;
  \item \_updatePartDetailFormDataHelper – помощник для работы с сервисом запросов базы данных.
\end{itemize}

Методы:
\begin{itemize}
  \item AddPaymentsRoutingDataAction() – конструктор класса;
  \item Task<IDataResult<PartDetailFormModel, PartDetailFormModel>> Proces"=sAsync(PartDetailFormModel contextModel) – метод, выполняющий обработку данных и запись этих данных в таблицу. На вход поступает модель, которую необходимо обновить, на выходе получается результат добавления.
\end{itemize}

\subsubsection{Класс AddSubComponentDataAction }
\label{sub:arch_and_mod:data_layer:add_sub_component}

представляет собой класс, реализующий добавление новых полей типа SubKeyComponent. При вызове данный класс строит объект JSON, в котором содержатся поля, необходимые для добавления новой записи. Является реализацией интерфейса IDataAction.

Поля:
\begin{itemize}
  \item \_applicationContext – поле контекста приложения, содержит в себе информацию о текущем пользователе, локализации;
  \item \_dataResultFactory – поле фабрики результатов, формирует модель результат валидации для дальнейшей передачи на клиент;
  \item \_dataSourceInvoker – поле исполнения процедуры в базе данных. На вход принимает модель запроса, на выходе возвращает результат выполнения процедуры;
  \item \_partDetailQuoteHelper – поле класса помощника, выполняющего проверку на разрешение редактирования записи, и контейнера констант;
  \item \_quoteWizardSettings – контейнер настроек для текущего пользователя;
  \item \_updatePartDetailFormDataHelper – помощник для работы с сервисом запросов базы данных.
\end{itemize}

Методы:
\begin{itemize}
  \item AddSubKeyComponentDataAction () – конструктор класса;
  \item Task<IDataResult<PartCostDetailFormModel, PartCostDetailFormModel>> Pro"=cessAsync(PartCostDetailFormModel contextModel) – метод, выполняющий обработку данных и запись этих данных в таблицу. На вход поступает модель, которую необходимо обновить, на выходе получается результат добавления.
\end{itemize}

\subsubsection{Класс AddSupplyItemDataAction }
\label{sub:arch_and_mod:data_layer:add_supply_item}

представляет собой класс, реализующий добавление новых полей типа SupplyItemsKey. При вызове данный класс строит объект JSON, в котором содержатся поля, необходимые для добавления новой записи. Является реализацией интерфейса IDataAction.

Поля:
\begin{itemize}
  \item \_applicationContext – поле контекста приложения, содержит в себе информацию о текущем пользователе, локализации;
  \item \_dataResultFactory – поле фабрики результатов, формирует модель результат валидации для дальнейшей передачи на клиент;
  \item \_dataSourceInvoker – поле исполнения процедуры в базе данных. На вход принимает модель запроса, на выходе возвращает результат выполнения процедуры;
  \item \_partCostDetailQuoteHelper – поле класса помощника, выполняющего проверку на разрешение редактирования записи, и контейнера констант;
  \item \_quoteWizardSettings – контейнер настроек для текущего пользователя;
  \item \_updatePartCostDetailFormDataHelper – помощник для работы с сервисом запросов базы данных.
\end{itemize}

Методы:
\begin{itemize}
  \item AddSupplyItemDataAction() – конструктор класса;
  \item Task<IDataResult<PartCostDetailFormModel, PartCostDetailFormModel>> Pro"=cessAsync(PartCostDetailFormModel contextModel) – метод, выполняющий обработку данных и запись этих данных в таблицу. На вход поступает модель, которую необходимо обновить, на выходе получается результат добавления.
\end{itemize}

\subsubsection{Класс CalculateRightSummaryTotalstDataAction }
\label{sub:arch_and_mod:data_layer:calculate_right_summary_totalst}

представляет собой класс, реализующий пересчет итогового значения прав на все группы, составляющих его хранилище. Является реализацией интерфейса IDataAc"=tion.

Поля:
\begin{itemize}
  \item \_applicationContext – поле контекста приложения, содержит в себе информацию о текущем пользователе, локализации;
  \item \_dataResultFactory – поле фабрики результатов, формирует модель результат валидации для дальнейшей передачи на клиент;
  \item \_dataSourceInvoker – поле исполнения процедуры в базе данных. На вход принимает модель запроса, на выходе возвращает результат выполнения процедуры. Является экземпляром сервиса доступа к базе данных. Данное поле инициализируется на этапе запуска приложения с помощью контейнера;
  \item \_partCostDetailQuoteHelper – поле класса помощника, выполняющего проверку на разрешение редактирования записи, и контейнера констант;
  \item \_quoteWizardSettings – контейнер настроек для текущего пользователя;
  \item \_updatePartCostDetailFormDataHelper – помощник для работы с сервисом запросов базы данных.
\end{itemize}

Методы:
\begin{itemize}
  \item CalculateRightSummaryTotalstDataAction() – конструктор класса;
  \item Task<IDataResult<PartDetailFormModel, PartDetailFormModel>> Proces"=sAsync(PartDetailFormModel contextModel) – метод, выполняющий обработку данных и запись этих данных в таблицу. На вход поступает модель, которую необходимо обновить, на выходе получается результат добавления.
\end{itemize}

\subsubsection{Класс CalculateProcessDataAction }
\label{sub:arch_and_mod:data_layer:calculate_process}

представляет собой класс, реализующий учет действий пользователя, который был сконфигурирован пользователем. Является реализацией интерфейса IDataAction.

Поля:
\begin{itemize}
  \item \_applicationContext – поле контекста приложения, содержит в себе информацию о текущем пользователе, локализации;
  \item \_dataResultFactory – поле фабрики результатов, формирует модель результат валидации для дальнейшей передачи на клиент;
  \item \_dataSourceInvoker – поле исполнения процедуры в базе данных. На вход принимает модель запроса, на выходе возвращает результат выполнения процедуры;
  \item \_partCostDetailQuoteHelper – поле класса помощника, выполняющего проверку на разрешение редактирования записи, и контейнера констант;
  \item \_quoteWizardSettings – контейнер настроек для текущего пользователя;
  \item \_updatePartCostDetailFormDataHelper – помощник для работы с сервисом запросов базы данных.
\end{itemize}

Методы:
\begin{itemize}
  \item CalculateProcessDataAction () – конструктор класса;
  \item Task<IDataResult<PartDetailFormModel, PartDetailFormModel>> Proces"=sAsync(PartDetailFormModel contextModel) – метод, выполняющий обработку данных и запись этих данных в таблицу. На вход поступает модель, которую необходимо обновить, на выходе получается результат добавления.
\end{itemize}

\subsubsection{Класс CalculateAccessesUserDataAction }
\label{sub:arch_and_mod:data_layer:calculate_accesses_user}

представляет собой класс, реализующий оценку прав пользователя на доступ к хранилищам. Является реализацией интерфейса IData"=Action.

Поля:
\begin{itemize}
  \item \_applicationContext – поле контекста приложения, содержит в себе информацию о текущем пользователе, локализации;
  \item \_dataResultFactory – поле фабрики результатов, формирует модель результат валидации для дальнейшей передачи на клиент;
  \item \_dataSourceInvoker – поле исполнения процедуры в базе данных. На вход принимает модель запроса, на выходе возвращает результат выполнения процедуры;
  \item \_partCostDetailQuoteHelper – поле класса помощника, выполняющего проверку на разрешение редактирования записи, и контейнера констант;
  \item \_quoteWizardSettings – контейнер настроек для текущего пользователя;
  \item \_updatePartCostDetailFormDataHelper – помощник для работы с сервисом запросов базы данных.
\end{itemize}

Методы:
\begin{itemize}
  \item CalculateProcessDataAction () – конструктор класса;
  \item Task<IDataResult<PartDetailFormModel, PartDetailFormModel>> Proces"=sAsync(PartDetailFormModel contextModel) – метод, выполняющий обработку данных и запись этих данных в таблицу. На вход поступает модель, которую необходимо обновить, на выходе получается результат добавления.
\end{itemize}

\subsubsection{Класс CheckProcessRoutingDeleteDataAction }
\label{sub:arch_and_mod:data_layer:check_process_routing_delete}

представляет собой класс, который проверяет возможность удаления записей с хранилища. Если процесс удалить не возможно, метод класса вернет валидационную ошибку с текстом сообщения. Иначе вернется результат успешной проверки, и запись будет удалена из базы данных. Является реализацией интерфейса IData"=Action.

Поля:
\begin{itemize}
  \item \_applicationContext – поле контекста приложения, содержит в себе информацию о текущем пользователе, локализации;
  \item \_dataResultFactory – поле фабрики результатов, формирует модель результат валидации для дальнейшей передачи на клиент;
  \item \_dataSourceInvoker – поле исполнения процедуры в базе данных. На вход принимает модель запроса, на выходе возвращает результат выполнения процедуры;
  \item \_partDetailQuoteHelper – поле класса помощника, выполняющего проверку на разрешение редактирования записи, и контейнера констант;
  \item \_quoteWizardSettings – контейнер настроек для текущего пользователя;
  \item \_updatePartDetailFormDataHelper – помощник для работы с сервисом запросов базы данных.
\end{itemize}

Методы:
\begin{itemize}
  \item CheckProcessRoutingDeleteDataAction () – конструктор класса;
  \item Task<IDataResult<PartDetailFormModel, PartDetailFormModel>> Proces"=sAsync(PartDetailFormModel contextModel) – метод, выполняющий проверку данных на наличие связей в базе данных. На вход поступает модель, которую необходимо проверить, на выходе получается валидационный результат;
\end{itemize}

\subsubsection{Класс GetPartDetailDataAction }
\label{sub:arch_and_mod:data_layer:get_part_detail}

представляет собой класс, который позволяет получить данные для модели на клиенте. Является реализацией интерфейса IDataAction.

Поля:
\begin{itemize}
  \item \_dataResultFactory – поле фабрики результатов, формирует модель результат валидации для дальнейшей передачи на клиент;
  \item \_partDetailQuoteHelper – поле класса помощника, выполняющего проверку на разрешение редактирования записи, и контейнера констант;
  \item \_getPartDetailFormDataHelper – помощник для работы с сервисом запросов базы данных для получения модели.
\end{itemize}

Методы:
\begin{itemize}
  \item GetPartDetailDataAction() – конструктор класса;
  \item Task<IDataResult<PartDetailFormModel, PartDetailFormModel>> Proces"=sAsync(PartDetailFormModel contextModel) – метод, выполняющий получающий данные для конкретной записи в базе данных. На вход поступает модель с ключами, на выходе получается модель типа Model.
\end{itemize}

\subsubsection{Класс ViewPartDetailFormAction }
\label{sub:arch_and_mod:data_layer:view_part_detail}

представляет собой главный класс, так называемую точку входа в приложение. Данный класс является фабрикой по формированию и наполнению модели данных на серверной стороне приложения. В приложении так же используются однотипные классы для формирования объектов отображения. Подобные классы являются реализацией интерфейса IViewAction. Единственным отличием между Managed"=Group и ManagedEntry является тип возвращаемой модели. Для ManagedGroup это ManagedGroupModel, для ManagedEntry – ManagedEntryModel.

Поля:
\begin{itemize}
  \item \_actionBarModelBuilderFactory – поле фабрики панели действий, формирует модель действий для дальнейшей передачи на клиент;
  \item \_partDetailQuoteHelper – поле класса помощника, выполняющего проверку на разрешение редактирования записи, и контейнера констант, выполняет роль контейнера, содержащего в себе текстовые константы и методы, которые необходимы для получения разрешений и настроек;
  \item \_getPartDetailFormDataHelper – помощник для работы с сервисом запросов базы данных для получения модели.
  \item \_ModelBuilder* – генератор классов-фабрик, которые возвращают конкретную модель для наполнения формы. В данном приложении используется два типа моделей: модели таблицы и модель формы, которая выступает контейнером для таблиц. Все такие классы принимают на вход модель данных, возвращают реализацию интерфейса ISectionViewModelBuilder.
\end{itemize}

Методы:
\begin{itemize}
  \item ViewPartDetailFormAction() – конструктор класса;
  \item Task<IDataResult<PartDetailFormModel, PartDetailFormModel>> \foreignlanguage{english}{ProcessAsync(PartDetailFormModel contextModel)} – метод, получающий данные для конкретной записи в базе данных, формирующий из этих данных контекстную модель, инициирующий модель отображения и наполняющий модель отображения секциями. На вход поступает модель с ключами, на выходе получается модель отображения;
\end{itemize}

\subsubsection{Класс GetPartDetailFormDataHelper }
\label{sub:arch_and_mod:data_layer:view_part_detail}

представляет собой класс, возвращающий данные хранилища по запросу пользователя.

Поля:
\begin{itemize}
  \item \_applicationContext – поле контекста приложения, содержит в себе информацию о текущем пользователе, локализации;
  \item \_dataSourceInvoker – поле исполнения процедуры в базе данных. На вход принимает модель запроса, на выходе возвращает результат выполнения процедуры;
  \item \_glossaryWordProvider – поле, позволяющее получить глосаризованное значение текстовых составляющих записей;
  \item \_quoteWizardSettings – контейнер настроек для текущего пользователя;
\end{itemize}

Методы:
\begin{itemize}
  \item GetPartDetailFormDataHelper () – конструктор класса;
  \item Task CalculateAccessUserModel contextModel) – метод калькуляции прав доступа к хранилищу;
  \item Task GetModelMarkupBreakdownData (Model contextModel) – метод получения данных для групп хранилища пользователя;
  \item Task GetLinkedInlineProcessRouting (Model contextModel) – метод метод получения данных для таблицы встроенных процессов;
  \item Task GetMarkupSummary (Model contextModel) – метод возвращающий реализацию объекта типа IMarkupData;
  \item Task GetPartDetailFormModelData (Model contextModel) – метод получения данных для формы, обладающий наибольшим приоритетом, вызывается в первую очередь;
  \item Task GetQuoteRaw(Model contextModel) – метод получения данных для таблицы с побочными полями текущей записи;
  \item Task CalculateMarkup (Model contextModel) – метод валидации, возвращающий оценочную стойкость пароля;
  \item Task GetGlossaryLabels (Model contextModel) – метод получения локализованных строковых литералов;
  \item Task GetPartDetailItems (Model contextModel) – метод получения данных для каждой записи текущей группы;
  \item Task GetPasswordRouting (Model contextModel) – метод получения паролей для конкретной записи;
  \item Task GetQuotePartComponents (Model contextModel) – метод составляющих компоненты в производственном цикле, вне виртуального событийного цикла слоя;
  \item Task SetEntrySummary(Model contextModel) – метод установки новой записи;
\end{itemize}

\subsubsection{Класс UpdatePartDetailFormDataHelper }
\label{sub:arch_and_mod:data_layer:view_part_detail}

представляет собой класс-сервис, обновляющий данные для страницы. Класс аналогичен предыдущему~\ref{sub:arch_and_mod:data_layer:view_part_detail}, за исключением того, что данные отправляются в базу, а не считываются.

Поля:
\begin{itemize}
  \item \_applicationContext – поле контекста приложения, содержит в себе информацию о текущем пользователе, локализации;
  \item \_dataSourceInvoker – поле исполнения процедуры в базе данных. На вход принимает модель запроса, на выходе возвращает результат выполнения процедуры;
  \item \_glossaryWordProvider – поле, позволяющее получить глосаризованное значение текстовых составляющих страницы;
  \item \_quoteWizardSettings – контейнер настроек текущего пользователя.
\end{itemize}

Методы:
\begin{itemize}
  \item UpdatePartDetailFormDataHelper() – конструктор класса;
  \item Task UpdateModelMarkupBreakdownData (Model contextModel) – метод обновления данных для групп хранилища пользователя;
  \item Task UpdateLinkedInlineProcessRouting (Model contextModel) – метод метод обновления данных для таблицы встроенных процессов;
  \item Task UpdateMarkupSummary (Model contextModel) – метод обновления реализации объекта типа IMarkupData;
  \item Task UpdatePartDetailFormModelData (Model contextModel) – метод обновления данных для формы, метод получения данных, обладающий наибольшим приоритетом, вызывается в первую очередь;
  \item Task UpdateQuoteRaw(Model contextModel) – метод обновления данных для таблицы с побочными полями текущей записи;
  \item Task UpdatePartDetailItems (Model contextModel) – метод обновления данных для каждой записи текущей группы;
  \item Task UpdatePasswordRouting (Model contextModel) – метод обновления процессов для конкретной записи;
  \item Task UpdateQuotePartComponents(Model contextModel) – метод обновления составляющих и компонент производственного процесса странице;
\end{itemize}

\subsection{Классы Application Server Layer}
\label{sub:arch_and_mod:application_server_layer}

К классам и методам сервисов относятся классы и методы, имеющие атрибуты @Path и принимающих в качестве параметра объект типа Request. Основной ролью данных классов и методов является формирования ответов для пользователя с целью дальнейшего взаимодействия через сервисы посредством AJAX. Методы сервиса являются основными методами влияния на среду выполнения приложения.

\subsubsection{Класс UserManagementRestResource }
\label{sub:arch_and_mod:application_server_layer:user-management-rest-resource}

используется для методов сервиса по управлению пользователями и группами. Добавлен атрибут @Path со значением /users. Является реализацией интерфейса IUserManage"=mentService.

Поля:
\begin{itemize}
  \item userUtil – поле типа UserUtil, хранящее менеджер пользователей приложения;
  \item groupManager – поле типа GroupManager, используется для логирования исключительных ситуаций, сбоев и отсутствие разрешений;
  \item logger – поле типа Logger, используется для логирования исключительных ситуаций, сбоев и отсутствие разрешений.
\end{itemize}

Методы:
\begin{itemize}
  \item UserManagementRestResource (UserUtil userUtil, GroupManager group"=Manager) – конструктор класса сервиса по управлению пользователями и группами, принимающего в качестве менеджер пользователей и менеджер групп;
  \item Response addUserToGroup(Request req) – метод сервиса, который добавляет пользователя к группе. Метод при успешном выполнении возвращает значение в формате JSON \lstinline!{"responseType": "Success", "message": ""}!.\\ Атрибут @Path имеет значение /addToGroup. Идентификаторы групп и пользователя передаются в качестве параметров HTTP запроса;
  \item Response moveUserFromGroup(Request req) – метод сервиса, который перемещает пользователя из одной группы в другую. Метод при успешном выполнении возвращает \lstinline!{"responseType": "Success", "message": ""}!. Атрибут @Path имеет значение /moveFromGroup. Идентификаторы групп и пользователя передаются в качестве параметров HTTP запроса;
  \item Response removeFromGroup(Request req) – метод сервиса, который удаляет пользователя из группы. Метод при успешном выполнении возвращает значение в формате JSON \lstinline!{"responseType": "Success", "message": ""}!. Атрибут @Path имеет значение /addToGroup. Идентификаторы групп и пользователя передаются в качестве параметров HTTP запроса;
  \item Response getUserInfo(Request req) – метод сервиса, который возвращает данные о конкретном пользователе. Метод при успешном выполнении возвращает сериализованный объект User в формате JSON . Атрибут @Path имеет значение /addToGroup. Идентификатор пользователя передаются в качестве параметра HTTP запроса.
\end{itemize}

\subsubsection{Класс LabelManagementRestResource }
\label{sub:arch_and_mod:application_server_layer:label_management_rest_resource}

используется для методов сервиса по управлению записями, группами и хранилищами. Добавлен атрибут @Path со значением /labels. Является реализацией интерфейса ILabel"=ManagementService.

Поля:
\begin{itemize}
  \item labelManager – поле типа LabelManager, хранящее менеджер записей приложения;
  \item issueManager – поле типа IssueManager, используется для логирования исключительных ситуаций, сбоев и отсутствие разрешений;
  \item logger – поле типа Logger, используется для логирования исключительных ситуаций, сбоев и отсутствие разрешений.
\end{itemize}

Методы:
\begin{itemize}
  \item LabelManagementAction(LabelManager labelManager, IssueManager \\issueManager) – конструктор класса сервиса по управлению записями, группами и хранилищами, менеджер записей и задач;
  \item Response addIssueToLabel(HttpServletRequest req) – метод сервиса, который добавляет группу к хранилищу. Метод при успешном выполнении возвращает значение в формате \lstinline!{"responseType": "Success", "message": ""}!.\\ Атрибут @Path имеет значение /addToLabel. Идентификатор задачи и значение метки передаются в качестве параметров HTTP запроса;
  \item Response moveIssueFromLabel(Request req) – метод сервиса, который удаляет записи с одной группы и добавляет к другой. Метод при успешном выполнении возвращает \lstinline!{"responseType": "Success", "message": ""}!. Атрибут @Path имеет значение /moveFromLabel. Идентификатор задачи и значение метки передаются в качестве параметров HTTP запроса;
  \item Response removeIssueFromLabel(Request req) – метод сервиса, который удаляет запись из группы. Метод при успешном выполнении возвращает значение в формате JSON \lstinline!{"responseType": "Success", "message": ""}!. Атрибут @Path имеет значение /removeFromLabel. Идентификатор задачи и значение метки передаются в качестве параметров HTTP запрос.
\end{itemize}

\subsubsection{Класс BaseResponse }
\label{sub:arch_and_mod:application_server_layer:base_response}

используются для взаимодействия с функциональностью клиента, сервисов и серверных элементов функциональных действий. Класс BaseResponse является базовым классом для любой сущности типа Application Server Layer.

Поля:
\begin{itemize}
  \item responseType – поле типа ResponseType, которое хранит статус ответа;
  \item message – поле строкового типа, которое используется для передачи сообщения об ошибке или для оповещения;
\end{itemize}

Методы:
\begin{itemize}
  \item BaseResponse() – конструктор класса базового ответа сервиса и функционального действия;
  \item BaseResponse(ResponseType responseType, String message) – перегрузка конструктора класса базового ответа сервиса и функционального действия с параметрами;
  \item getMessage() и setMessage(String message) – методы для получения и установки значения сообщения;
  \item getResponseType() и setResponseType(ResponseType responseType) – методы для получения и установки типа ответа сервиса и функционального действия.
\end{itemize}

\subsubsection{Класс Quickmanagement }
\label{sub:arch_and_mod:application_server_layer:quickmanagement}

является классом связи с сервисом и определяет события подписки на запросы пользователя.

Методы:
\begin{itemize}
  \item Quickmanagement addIssueToLabel(Array<string> ...options, Function callback) – метод асинхронного вызова метода addIssueToLabel сервиса Label"=ManagementService. Параметр options должен содержать идентификатор записи, идентификатор группы. В качестве значения параметра callback передается функция, которая будет вызвана после возврата ответа сервиса. Функция callback принимает в качестве параметра объект класса BaseResponse;
  \item Quickmanagement moveIssueFromLabel(Array<string> ...options, Fun"=ction callback) – метод асинхронного вызова метода moveIssueFromLabel сервиса LabelManagementService. Параметр options должен содержать идентификатор задачи, идентификаторы метки назначения и метки источника. В качестве значения параметра callback передается функция, которая будет вызвана после возврата ответа сервиса. Функция callback принимает в качестве параметра объект класса BaseResponse;
  \item Quickmanagement removeIssueFromLabel(Array<string> ...options, Fun"=ction callback) – метод асинхронного вызова метода removeIssueFromLabel сервиса LabelManagementService. Параметр options должен содержать идентификатор задачи, идентификаторы метки, из которой удаляется задача. В качестве значения параметра callback передается функция, которая будет вызвана после возврата ответа сервиса. Функция callback принимает в качестве параметра объект класса BaseResponse;
  \item Quickmanagement addUserToGroup(Array<string> ...options, Function callback) – метод асинхронного вызова метода addUserToGroup сервиса User"=ManagementService. Параметр options должен содержать идентификатор пользователя, идентификатор группы. В качестве значения параметра callback передается функция, которая будет вызвана после возврата ответа сервиса. Функция callback принимает в качестве параметра объект класса BaseResponse;
  \item Quickmanagement moveUserFromGroup(Array<string> ...options, Func"=tion callback) – метод асинхронного вызова метода moveUserFromGroup сервиса UserManagementService. Параметр options должен содержать идентификатор пользователя, идентификаторы группы назначения и группы источника. В качестве значения параметра callback передается функция, которая будет вызвана после возврата ответа сервиса. Функция callback принимает в качестве параметра объект класса BaseResponse;
  \item Quickmanagement removeFromGroup(Array<string> ...options, Functi"=on callback) – метод асинхронного вызова метода removeFromGroup сервиса UserManagementService. Параметр options должен содержать идентификатор пользователя, идентификатор пользователя, который удаляется из группы. В качестве значения параметра callback передается функция, которая будет вызвана после возврата ответа от сервиса. Функция callback принимает в качестве параметра объект класса BaseResponse;
  \item Quickmanagement getUserInfo(Array<string> ...options, Function call"=back) – метод асинхронного вызова метода getUserInfo сервиса UserManage"=mentService. Параметр options должен содержать идентификатор пользователя В качестве значения параметра callback передается функция, которая будет вызвана после возврата ответа от сервиса. Функция callback принимает в качестве параметра объект класса User;
  \item T parseResponse(Response response) – метод обработки ответа сервиса. Параметр response имеет значение ответа сервиса.
\end{itemize}