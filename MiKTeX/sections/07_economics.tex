\newcommand{\byn}{\text{руб.}}

\newcommand{\totalloc}{\text{V}_\text{о}}
\newcommand{\normativelaboriousness}{\text{Т}_\text{н}}
\newcommand{\complexity}{\text{К}_\text{с}}
\newcommand{\stdmodules}{\text{К}_\text{т}}
\newcommand{\novelty}{\text{К}_\text{н}}
\newcommand{\totallaboriousness}{\text{Т}_\text{о}}

\newcommand{\effectivetimefund}{\text{Ф}_\text{эф}}
\newcommand{\daysinyear}{\text{Д}_\text{г}}
\newcommand{\holidays}{\text{Д}_\text{п}}
\newcommand{\weekends}{\text{Д}_\text{в}}
\newcommand{\vacationdays}{\text{Д}_\text{о}}

\newcommand{\developersnumber}{\text{Ч}_\text{р}}
\newcommand{\developmenttime}{\text{Т}_\text{р}}

\newcommand{\firstratetariffsymbol}{\text{Т}_\text{ч}^1}
\newcommand{\averagehourspermonthsymbol}{\text{Ф}_\text{р}}
\newcommand{\hourspershiftsymbol}{\text{Т}_\text{ч}}
\newcommand{\bonusratesymbol}{K}

\newcommand{\basewagesymbol}{\text{З}_\text{о}}
\newcommand{\additionalwageratesymbol}{\text{Н}_\text{д}}
\newcommand{\additionalwagesymbol}{\text{З}_\text{д}}

\newcommand{\ssfratesymbol}{\text{Н}_\text{сз}}
\newcommand{\ssfchargessymbol}{\text{З}_\text{сз}}
\newcommand{\insuranceratesymbol}{\text{Н}_\text{ос}}
\newcommand{\insurancechargessymbol}{\text{З}_\text{ос}}


\FPeval{\totalProgramSize}{28940}

\FPeval{\normativeManDays}{520}

\FPeval{\additionalComplexity}{clip(0.06+0.07)}

\FPeval{\stdModuleUsageFactor}{0.7}
\FPeval{\noveltyFactor}{0.9}

\FPeval{\daysInYear}{365}
\FPeval{\redLettersDaysInYear}{9}
\FPeval{\weekendDaysInYear}{103}
\FPeval{\vacationDaysInYear}{21}

\FPeval{\firstratetariff}{265}
\FPeval{\averagehourspermonth}{168.3}
\FPeval{\hourspershift}{8}
\FPeval{\bonusrate}{1.5}
\FPeval{\additionalwagerate}{20}
\FPeval{\ssfrate}{34}
\FPeval{\insurancerate}{0.6}


\FPdebugtrue

\FPeval{\totalProgramSizeCorrected}{\totalProgramSize}
\newcommand{\totallocfactor}{\num{\totalProgramSizeCorrected}}

\newcommand{\additionalcomplexityfactor}{\num{\additionalComplexity}}
\FPeval{\complexityFactor}{clip(1 + \additionalComplexity)}
\newcommand{\complexityfactor}{\num{\complexityFactor}}

\newcommand{\stdmodulesfactor}{\num{\stdModuleUsageFactor}}
\newcommand{\noveltyfactor}{\num{\noveltyFactor}}

\FPeval{\adjustedManDaysExact}{clip( \normativeManDays * \complexityFactor * \stdModuleUsageFactor * \noveltyFactor )}
\FPround{\adjustedManDays}{\adjustedManDaysExact}{0}
\newcommand{\normativelaboriousnessfactor}{\num{\normativeManDays}}
\newcommand{\totallaboriousnessfactor}{\num{\adjustedManDays}}


\FPeval{\workingDaysInYear}{ clip( \daysInYear - \redLettersDaysInYear - \weekendDaysInYear - \vacationDaysInYear ) }
\newcommand{\daysinyearfactor}{\num{\daysInYear}}
\newcommand{\holidaysfactor}{\num{\redLettersDaysInYear}}
\newcommand{\weekendsfactor}{\num{\weekendDaysInYear}}
\newcommand{\vacationdaysfactor}{\num{\vacationDaysInYear}}
\newcommand{\effectivetimefundfactor}{\num{\workingDaysInYear}}

\FPeval{\requiredNumberOfProgrammers}{3}
\newcommand{\developersnumberfactor}{\num{\requiredNumberOfProgrammers}}
\FPeval{\developmentTimeYearsExact}{\adjustedManDays / (\requiredNumberOfProgrammers * \workingDaysInYear)}
\FPround{\developmentTimeYears}{\developmentTimeYearsExact}{2}
\newcommand{\developmenttimeyearsfactor}{\num{\developmentTimeYears}}
\FPeval{\developmentTimeMonthsExact}{\developmentTimeYearsExact * 12}
\FPround{\developmentTimeMonths}{\developmentTimeMonthsExact}{1}
\newcommand{\developmenttimemonthsfactor}{\num{\developmentTimeMonths}}
\FPeval{\developmentTimeDaysExact}{\developmentTimeYearsExact * \daysInYear}
\FPround{\developmentTimeDays}{\developmentTimeDaysExact}{0}
\newcommand{\developmenttimefactor}{\num{\developmentTimeDays}}


\FPeval{\employeeAMonthExact}{\firstratetariff * 3.98}
\FPeval{\employeeBMonthExact}{\firstratetariff * 3.48}
\FPeval{\employeeCMonthExact}{\firstratetariff * 2.65}
\FPeval{\employeeAHourExact}{\employeeAMonthExact / \averagehourspermonth}
\FPeval{\employeeBHourExact}{\employeeBMonthExact / \averagehourspermonth}
\FPeval{\employeeCHourExact}{\employeeCMonthExact / \averagehourspermonth}
\FPround{\employeeAMonth}{\employeeAMonthExact}{2}
\FPround{\employeeBMonth}{\employeeBMonthExact}{2}
\FPround{\employeeCMonth}{\employeeCMonthExact}{2}
\FPround{\employeeAHour}{\employeeAHourExact}{2}
\FPround{\employeeBHour}{\employeeBHourExact}{2}
\FPround{\employeeCHour}{\employeeCHourExact}{2}
\newcommand{\employeeamonthwage}{\num{\employeeAMonth}}
\newcommand{\employeebmonthwage}{\num{\employeeBMonth}}
\newcommand{\employeecmonthwage}{\num{\employeeCMonth}}
\newcommand{\employeeahourwage}{\num{\employeeAHour}}
\newcommand{\employeebhourwage}{\num{\employeeBHour}}
\newcommand{\employeechourwage}{\num{\employeeCHour}}

\newcommand{\firstratetariffvalue}{\num{\firstratetariff}}
\newcommand{\averagehourspermonthvalue}{\num{\averagehourspermonth}}
\newcommand{\hourspershiftvalue}{\num{\hourspershift}}
\newcommand{\bonusratevalue}{\num{\bonusrate}}

\FPeval{\basewageExact}{(\employeeAHour + \employeeBHour + \employeeCHour) * \hourspershift * \developmentTimeDays * \bonusrate}
\FPround{\basewage}{\basewageExact}{2}
\newcommand{\basewagevalue}{\num{\basewage}}

\newcommand{\additionalwageratevalue}{\num{\additionalwagerate}\%}
\FPeval{\additionalwage}{round(\basewage * \additionalwagerate / 100, 2)}
\newcommand{\additionalwagevalue}{\num{\additionalwage}~\byn}

\FPeval{\ssfcharges}{round((\basewage + \additioanlwage) * \ssfrate / 100, 2)}
\FPeval{\insurancecharges}{round((\basewage + \additioanlwage) * \insurancerate / 100, 2)}
\newcommand{\ssfchargesvalue}{\num{\ssfcharges}~\byn}
\newcommand{\ssfratevalue}{\num{\ssfrate}\%}
\newcommand{\insurancechargesvalue}{\num{\insurancecharges}~\byn}
\newcommand{\insuranceratevalue}{\num{\insurancerate}\%}


% граница выполнения: дальше недоделано


\FPeval{\employmentFstExact}{clip( \adjustedManDays / \requiredNumberOfProgrammers )}
\FPtrunc{\employmentFst}{\employmentFstExact}{0}

\FPeval{\employmentSnd}{clip(\adjustedManDays - \employmentFst)}


\FPeval{\workingHoursInMonth}{160}
\FPeval{\salaryPerHourFstExact}{clip( \tariffRateFirst * \tariffFactorFst / \workingHoursInMonth )}
\FPeval{\salaryPerHourSndExact}{clip( \tariffRateFirst * \tariffFactorSnd / \workingHoursInMonth )}
\FPround{\salaryPerHourFst}{\salaryPerHourFstExact}{0}
\FPround{\salaryPerHourSnd}{\salaryPerHourSndExact}{0}

\FPeval{\bonusRate}{1.5}
\FPeval{\workingHoursInDay}{8}
\FPeval{\totalSalaryExact}{clip( \workingHoursInDay * \bonusRate * ( \salaryPerHourFst * \employmentFst + \salaryPerHourSnd * \employmentSnd ) )}
\FPround{\totalSalary}{\totalSalaryExact}{0}

\FPeval{\additionalSalaryNormative}{20}

\FPeval{\additionalSalaryExact}{clip( \totalSalary * \additionalSalaryNormative / 100 )}
\FPround{\additionalSalary}{\additionalSalaryExact}{0}

\FPeval{\socialNeedsNormative}{0.5}
\FPeval{\socialProtectionNormative}{34}
\FPeval{\socialProtectionFund}{ clip(\socialNeedsNormative + \socialProtectionNormative) }

\FPeval{\socialProtectionCostExact}{clip( (\totalSalary + \additionalSalary) * \socialProtectionFund / 100 )}
\FPround{\socialProtectionCost}{\socialProtectionCostExact}{0}

\FPeval{\taxWorkProtNormative}{4}
\FPeval{\taxWorkProtCostExact}{clip( (\totalSalary + \additionalSalary) * \taxWorkProtNormative / 100 )}
\FPround{\taxWorkProtCost}{\taxWorkProtCostExact}{0}
\FPeval{\taxWorkProtCost}{0} % это считать не нужно, зануляем чтобы не менять формулы

\FPeval{\stuffNormative}{3}
\FPeval{\stuffCostExact}{clip( \totalSalary * \stuffNormative / 100 )}
\FPeval{\stuffCost}{\stuffCostExact}

\FPeval{\timeToDebugCodeNormative}{15}
\FPeval{\reducingTimeToDebugFactor}{0.3}
\FPeval{\adjustedTimeToDebugCodeNormative}{ clip( \timeToDebugCodeNormative * \reducingTimeToDebugFactor ) }

\FPeval{\oneHourMachineTimeCost}{5000}

\FPeval{\machineTimeCostExact}{ clip( \oneHourMachineTimeCost * \totalProgramSizeCorrected / 100 * \adjustedTimeToDebugCodeNormative ) }
\FPround{\machineTimeCost}{\machineTimeCostExact}{0}

\FPeval{\businessTripNormative}{15}
\FPeval{\businessTripCostExact}{ clip( \totalSalary * \businessTripNormative / 100 ) }
\FPround{\businessTripCost}{\businessTripCostExact}{0}

\FPeval{\otherCostNormative}{20}
\FPeval{\otherCostExact}{clip( \totalSalary * \otherCostNormative / 100 )}
\FPround{\otherCost}{\otherCostExact}{0}

\FPeval{\overheadCostNormative}{100}
\FPeval{\overallCostExact}{clip( \totalSalary * \overheadCostNormative / 100 )}
\FPround{\overheadCost}{\overallCostExact}{0}

\FPeval{\overallCost}{clip( \totalSalary + \additionalSalary + \socialProtectionCost + \taxWorkProtCost + \stuffCost + \machineTimeCost + \businessTripCost + \otherCost + \overheadCost ) }

\FPeval{\supportNormative}{30}
\FPeval{\softwareSupportCostExact}{clip( \overallCost * \supportNormative / 100 )}
\FPround{\softwareSupportCost}{\softwareSupportCostExact}{0}


\FPeval{\baseCost}{ clip( \overallCost + \softwareSupportCost ) }

\FPeval{\profitability}{35}
\FPeval{\incomeExact}{clip( \baseCost / 100 * \profitability )}
\FPround{\income}{\incomeExact}{0}

\FPeval{\estimatedPrice}{clip( \income + \baseCost )}

\FPeval{\localRepubTaxNormative}{3.9}
\FPeval{\localRepubTaxExact}{clip( \estimatedPrice * \localRepubTaxNormative / (100 - \localRepubTaxNormative) )}
\FPround{\localRepubTax}{\localRepubTaxExact}{0}
\FPeval{\localRepubTax}{0}

\FPeval{\ndsNormative}{20}
\FPeval{\ndsExact}{clip( (\estimatedPrice + \localRepubTax) / 100 * \ndsNormative )}
\FPround{\nds}{\ndsExact}{0}


\FPeval{\sellingPrice}{clip( \estimatedPrice + \localRepubTax + \nds )}

\FPeval{\taxForIncome}{18}
\FPeval{\incomeWithTaxes}{clip(\income * (1 - \taxForIncome / 100))}
\FPround\incomeWithTaxes{\incomeWithTaxes}{0}

\section{Технико-экономическое обоснование разработки и внедрения программного средства}
\label{sec:economics}

\subsection{Характеристика программного средства}
\label{sec:economics:description}

...

Разработки проектов программных средств связана со значительными затратами ресурсов. В связи с этим создание и реализация каждого проекта программного обеспечения нуждается в соответствующем технико-экономическом обосновании~\cite{palitsyn}, которое и описывается в данном разделе.

\subsection{Определение объема и трудоемкости ПС}
\label{sec:economics:labouriousness}

Целесообразность создания ПС требует проведения предварительной экономической оценки. Экономический эффект у разработчика ПС зависит от объема инвестиций в разработку проекта, цены на готовый продукт и количества проданных копий, и проявляется в виде роста чистой прибыли. 

Оценка стоимости создания ПС со стороны разработчика предполагает составление сметы затрат, вычисление цены и прибыли от реализации разрабатываемого программного средства. 

Исходные данные, которые будут использоваться при расчете сметы затрат, представлены в таблице~\ref{table:economics:labouriousness:initial_data}.

\begin{table}[!ht]
\caption{Исходные данные}
\label{table:economics:labouriousness:initial_data}
\centering
	\begin{tabular}{{ 
	|>{\raggedright}m{0.6\textwidth} | 
	 >{\centering}m{0.17\textwidth} | 
	 >{\centering\arraybackslash}m{0.15\textwidth}|}}

  	\hline
	{\begin{center} Наименование показателя \end{center}} & Условное обозначение &	Значение \\
  
	\hline
	Категория сложности & & 3 \\

	\hline
	Дополнительный коэффициент сложности & $\sum\limits_{i=1}^{n} \text{К}_i$ & \additionalcomplexityfactor \\

	\hline
	Степень охвата функций стандартными модулями & $\stdmodules$ & \stdmodulesfactor \\

	\hline
	Коэффициент новизны & $\novelty$ & \noveltyfactor \\

	\hline
	Количество дней в году & $\daysinyear$ & \daysinyearfactor \\

	\hline
	Количество праздничных дней в году & $\holidays$ & \holidaysfactor \\

	\hline
	Количество выходных дней в году & $\weekends$ & \weekendsfactor \\

	\hline
	Количество дней отпуска & $\vacationdays$ & \vacationdaysfactor \\

	\hline
	Количество разработчиков & $\developersnumber$ & \developersnumberfactor \\

	\hline
	Тарифная ставка первого разряда, \byn & $\firstratetariffsymbol$ & \firstratetariffvalue \\

	\hline
	Среднемесячная норма рабочего времени, ч. & $\averagehourspermonthsymbol$ & \averagehourspermonthvalue \\

	\hline
	Продолжительность рабочей смены, ч. & $\hourspershiftsymbol$ & \hourspershiftvalue \\

	\hline
	Коэффициент премирования & $\bonusratesymbol$ & \bonusratevalue \\

	\hline
	Норматив дополнительной заработной платы & $\bonusratesymbol$ & \bonusratevalue \\

	\hline
	Норматив отчислений в ФСЗН & $\ssfratesymbol $ & \ssfratevalue \\

	\hline
	Норматив отчислений по обязательному страхованию & $\insuranceratesymbol $ & \insuranceratevalue \\

	\hline
	\end{tabular}
\end{table}

Перед определением сметы затрат на разработку программного средства необходимо определить его объём. Однако, на стадии ТЭО нет возможности рассчитать точные объемы функций, вместо этого с помощью применения действующих нормативов рассчитываются прогнозные оценки. В качестве метрики измерения объема программных средств используется строка их исходного кода (LOC -- lines of code). Данная метрика широко распространена, поскольку она непосредственно связана с конечным продуктом, может применяться от на всём протяжении проекта и, кроме того, может применяться для сопоставления размеров программного обеспечения. Далее под строкой исходного кода будем понимать количество исполняемых операторов.

Расчет объема функций программного средства и общего объема приведен в таблице~\ref{table:economics:labouriousness:function_sizes}. 

\begin{table}[!ht]
\caption{Перечень и объём функций программного модуля}
\label{table:economics:labouriousness:function_sizes}
\centering
	\begin{tabular}{{ | >{\centering}m{0.12\textwidth} | 
	>{\raggedright}m{0.6\textwidth} | 
	>{\centering\arraybackslash}m{0.2\textwidth}|}}

  	\hline
	\No{} функции & 
	{\begin{center} Наименование (содержание) \end{center}} & 
	Объём функции, LoC \\
  
	\hline 
	101 & Организация ввода информации & \num{100} \\

	\hline
	102 & Контроль, предварительная обработка и ввод информации & \num{500} \\

	\hline
	109 & Организация ввода/вывода информации в интерактивном режиме & \num{190} \\

	\hline
	111 & Управление вводом/выводом & \num{2600} \\

	\hline
	204 & Обработка наборов и записей базы данных & \num{1900} \\

	\hline
	207 & Манипулирование данными & \num{8000} \\

	\hline
	208 & Организация поиска и поиск в БД & \num{7500} \\

	\hline
	304 & Обслуживание файлов & \num{500} \\

	\hline
	305 & Обработка файлов & \num{800} \\

	\hline
	309 & Формирование файла & \num{1000} \\

	\hline
	506 & Обработка ошибочных и сбойных ситуаций & \num{500} \\

	\hline
	507 & Обеспечение интерфейса между компонентами & \num{750} \\

	\hline
	601 & Отладка прикладных программ в интерактивном режиме & \num{4300} \\

	\hline
	707 & Графический вывод результатов & \num{300} \\

	\hline
	 & Общий объем & \totallocfactor \\

	\hline
	\end{tabular}
\end{table}

Исходя из определенной 3-ей категории сложности и общего объема ПС $\totalloc = \totallocfactor$, нормативная трудоемкость $\normativelaboriousness = \normativelaboriousnessfactor~\text{чел.д.}$~\cite[приложение 3]{palitsyn}. Перед определением общей трудоемкости разработки необходимо определить несколько коэффициентов.

Коэффициент сложности, который учитывает дополнительные затраты труда, связанные с обеспечением интерактивного доступа и хранения, и поиска данных в сложных структурах~\cite[приложение 4, таблица П.4.2]{palitsyn}

\begin{equation}
	\complexity = 1 + \sum_{i=1}^{n} \text{К}_i = 1 + \num{0.06} + \num{0.07} = \complexityfactor,
\end{equation}
\begin{explanation}
где & $ \text{К}_i $ & коэффициент, соответствующий степени повышения сложности за счет конкретной характеристики;\\
& $ n $ & количество учитываемых характеристик.
\end{explanation}

Коэффициент $\stdmodules$, учитывающий степень использования при разработке стандартных модулей, для разрабатываемого приложения, в котором степень охвата планируется на уровне около 50\%, примем равным \num{0.7}~\cite[приложение 4, таблица П.4.5]{palitsyn}.

Коэффициент новизны разрабатываемого программного средства $\novelty$ примем равным \noveltyfactor, так как разрабатываемом программное средство принадлежит определенному параметрическому ряду существующих программных средств~\cite[приложение 4, таблица П.4.4]{palitsyn}.

Исходя из выбранных коэффициентов, общая трудоемкость разработки $ \totallaboriousness = \normativelaboriousness \cdot \complexity \cdot \stdmodules \cdot \novelty = \normativelaboriousnessfactor \cdot \complexityfactor \cdot \stdmodulesfactor \cdot \noveltyfactor = \totallaboriousnessfactor~\text{чел.д.}$

Для расчета срока разработки проекта примем число разработчиков $\developersnumber = \developersnumberfactor$. Исходя из комментария к постановлению Министерства труда и социальной защиты Республики Беларусь от 05.10.16 №54 <<Об установлении
расчетной нормы рабочего времени на 2017 год>>~\cite{labour_calendar}, эффективный фонд времени работы одного человека составит
\begin{equation}
	\effectivetimefund = \daysinyear - \holidays - \weekends - \vacationdays = \num{\daysInYear} - \num{\redLettersDaysInYear} - \num{\weekendDaysInYear} - \num{\vacationDaysInYear} = \num{\workingDaysInYear}~\text{д.},
\end{equation}
\begin{explanation}
где & $ \text{Д}_\text{г} $ & количество дней в году;\\
& $ \text{Д}_\text{п} $ & количество праздничных дней в году;\\
& $ \text{Д}_\text{в} $ & количество выходных дней в году;\\
& $ \text{Д}_\text{о} $ & количество дней отпуска.
\end{explanation}

Тогда срок разработки проекта

\begin{equation}
	\developmenttime = \frac{\totallaboriousness}{\developersnumber \cdot \effectivetimefund} = \frac{\totallaboriousnessfactor}{\developersnumberfactor \cdot \effectivetimefundfactor} = \developmenttimeyearsfactor~\text{г.} = \developmenttimefactor~\text{д.}
\end{equation}

\subsection{Расчет сметы затрат}
\label{sec:economics:estimate}

Основной статьей расходов на создание ПО является заработная плата разработчиков проекта. Информация об исполнителях перечислена в таблице~\ref{table:economics:estimate:employees}. Кроме того, в таблице приведены данные об их тарифных разрядах, приведены разрядные коэффициенты, а также по формулам~\ref{eq:economics:estimate:month_wage} и~\ref{eq:economics:estimate:hour_wage} рассчитаны месячный и часовой оклады.

\begin{equation}
\label{eq:economics:estimate:month_wage}
	\text{T}_\text{м} = \text{T}_\text{м}^1 \cdot \text{T}_\text{к},
\end{equation}
\begin{equation}
\label{eq:economics:estimate:hour_wage}
	\text{T}_\text{ч} = \frac{\text{T}_\text{м}}{\text{Ф}_\text{р}},
\end{equation}
\begin{explanation}
где & $ \text{T}_\text{м} $ & месячный оклад;\\
	& $ \text{T}_\text{м}^1 $ & тарифная ставка 1-го разряда (положим ее равной \num{\firstratetariff} \byn);\\
	& $ \text{T}_\text{к} $ & тарифный коэффициент;\\
	& $ \text{T}_\text{ч} $ & часовой оклад;\\
	& $ \text{Ф}_\text{р} $ & среднемесячная норма рабочего времени (в 2017 г. составляет \num{168.3} ч.~\cite{labour_calendar}).
\end{explanation}

\begin{table}[!ht]
  \caption{Работники, занятые в проекте}
  \label{table:economics:estimate:employees}
  \begin{tabular}{| >{\raggedright}m{0.3\textwidth} 
                  | >{\centering}m{0.09\textwidth} 
                  | >{\centering}m{0.18\textwidth} 
                  | >{\centering}m{0.15\textwidth} 
                  | >{\centering\arraybackslash}m{0.15\textwidth}|}
	\hline
	{\begin{center}Исполнители\end{center}} & Разряд & Тарифный коэффициент & Месячный оклад, \byn & Часовой оклад, \byn \\

	\hline
	Руководитель проекта & 17 & \num{3.98} & \employeeamonthwage & \employeeahourwage \\

	\hline
	Ведущий инженер-программист & 15 & \num{3.48} & \employeebmonthwage & \employeebhourwage\\

	\hline
	Инженер-программист II категории & 11 & \num{2.65} & \employeecmonthwage & \employeechourwage\\
	\hline
  \end{tabular}
\end{table}

Тогда основная заработная плата исполнителей составит
\begin{equation}
\begin{align}
	\basewagesymbol &= \sum_{i=1}^n \text{Т}_\text{чi} \cdot \text{Т}_\text{ч} \cdot \text{Ф}_\text{п} \cdot K = \\
	&= (\employeeahourwage + \employeebhourwage + \employeechourwage) \cdot \hourspershiftvalue \cdot \developmenttimefactor \cdot \bonusratevalue = \basewagevalue~\text{\byn},
\end{align}
\end{equation}
\begin{explanation}
где & $ \text{Т}_\text{чi} $ & часовая тарифная ставка i-го исполнителя, \byn;\\
	& $ \text{Т}_\text{ч} $ & количество часов работы в день;\\
	& $ \text{Ф}_\text{п} $ & плановый фонд рабочего времени i-го исполнителя, д.;\\
	& $ K $ & коэффициент премирования (принятый равным \bonusratevalue).
\end{explanation}

Дополнительная заработная плата включает выплаты, предусмотренные законодательство о труде: оплата отпусков, льготных часов, времени  выполнения  государственных обязанностей и других выплат, не связанных с основной деятельностью исполнителей, и определяется по нормативу, установленному в организации, в процентах к основной заработной плате.
Приняв данный норматив $\additionalwageratesymbol = \additionalwageratevalue$, рассчитаем дополнительные выплаты

\begin{equation}
	\additionalwagesymbol = \frac{\basewagesymbol \cdot \additionalwageratesymbol}{100\%} = \frac{\basewagevalue \cdot \additionalwageratevalue}{100\%} = \additionalwagevalue
\end{equation}

Отчисления в фонд социальной защиты населения и в фонд обязательного страхования определяются в соответствии с действующим законодательством по нормативу в процентном отношении к фонду основной и дополнительной зарплат. 
